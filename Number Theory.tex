\documentclass[a4paper]{article}

\usepackage{amsmath}
\usepackage{amsfonts}
\usepackage{amssymb}
\usepackage{enumitem}
\setlength{\parindent}{0em}
\setlength{\parskip}{1em}
\renewcommand{\baselinestretch}{1}
\title{Number Theory: Lecture Notes}
\author{Anthony Dunford \and Chris Nash}
\begin{document}

\maketitle

\tableofcontents

% Lecture 1

\section{Divisibility and Primes}

\subsection{Introduction}

\textbf{Well ordering Principle:}\\
Let $S\neq0$ be a set of positive integers.\\
Then there exists $s\in S$ such that for all $a\in S, s\leq a$

\textbf{Induction:}\\
If a set $s$ of positive integers contains the integer $1$\\
And contains $n+1$ whenever it contains $n$\\
Then $S$ consists of all the positive integers

\subsection{Divisibility}

\textbf{Definition 1.1:} Divisibility\\
An integer $b$ is divisible by an integer $a\neq0$ if there is an integer $x$ such that $b=ax$.\\
We write $a|b$ (a divides b)

\textbf{Theorem 1.1:} Properties of divisibility
\begin{enumerate}
    \item $a|b\            \rightarrow a|bc\quad c\in \mathbb{Z}$
    \item $a|b\ \&\ b|c    \rightarrow a|c$
    \item $a|b\ \&\ a|c    \rightarrow a|(bx+cy)\quad x,y\in\mathbb{Z}$
    \item $a|b\ \&\ b|a    \rightarrow a=\pm b$

% Lecture 2

    \item $a|b,\ a>0,\ b>0 \rightarrow a\leq b$
    \item $m\neq0,\ a|b\   \leftrightarrow ma|mb$
\end{enumerate}

\textbf{Proof:} Theorem 1.1 (3)

$a|b \rightarrow b=ar$ for some $r\in \mathbb{Z}$ and $a|c \rightarrow c=as$ for some $s\in \mathbb{Z}$. Hence $bx+cy=a(rx+sy)$ and this proves that $a|(bx+cy)$


\textbf{Theorem 1.2:} The Division Algorithm\\
Let $a,b\in\mathbb{Z},\ a>0$.\\
Then there exists unique $q,r\in\mathbb{Z}$ such that $b=qa+r,\ 0\leq r<a$.\\
If $a\nmid b$ then $0<r<a$

\textbf{Proof:} Theorem 1.2

Consider the arithmetic progression:

$...,b-3a,b-2a,b-a,b,b+a,b+2a,b+3a,...$

In the sequence select the smallest non-negative member and denote it by $r$. Thus by definition $r$ satisfies the inequalities of the Theorem. But also $r$, being in the sequence, is of the form $b-qa$, and thus q is defined in terms of $r$.

To prove uniqueness we suppose there is another pair $q_1$ and $r_1$ satisfying the same conditions. First we prove that $r=r_1$. If not, we may presume that $r<r_1$ so that $0<r_1-r<a$ and then we see that $r_1-r=a(q-q_1)$ and so $a|(r_1-r)$, a contradiction to Theorem 1.1 (5). Hence $r=r_1$ and also $q=q_1$.

Note: We stated the Theorem with $a>0$. However this is not necessary and we may formulate as:

Given $a,b\in\mathbb{Z}$ , $a\neq 0$ , there exists $q,r\in\mathbb{Z}$ such that $b=qa+r$ , $0\leq r < |a|$.


\textbf{Definition 1.2:}\\
The integer $a$ is a \underline{common divisor} of $b$ and $c$ if $a|b$, $a|c$ and at least $b\neq0$ or $c\neq0$, the greatest among their common divisors is called the \underline{greatest common divisor} of $b$ and $c$ and is denoted by $gcd(b,c)$ or $(b,c)$.

Let $b_1,...,b_n\in\mathbb{Z}$, not all zero. We denote $g=(b_1,...b_n)$ to be the greatest common divisor.

\textbf{Theorem 1.3:}\\
If $g=(b,c)$, then there exist $x_0,y_0\in\mathbb{Z}$ such that $g=(b,c)=bx_0+cy_0$

\textbf{Proof:} Theorem 1.3\\
Consider the linear combination $bx+cy$, where $x,y$ range over all the integers. This set of integers \{$bx+cy$\} includes positive and negative values and also 0. ($x=y=0$). Choose $x_0$ and $y_0$ so that $bx_0+cy_0$ is the least positive integer $l$ in the set. Thus $l=bx_0+cy_0$.

Next we prove that $l|b$ and $l|c$. Assume that $l \nmid b$ , then it follows that there exists integers $q$ and $r$ , by Theorem 1.2, such that $b=lq+r$ with $0<r<l$. Hence we have $r=b-lq=b-q(bx_0+cy_0)=b(1-qx_0)+c(-qy_0)$, and thus $r$ is in the set \{$bx+cy$\}. This contradicts the fact that $l$ is the least positive integer in \{$bx+cy\}$. Similar proof for $l|c$. Now since $g=(b,c)$ we may write $b=gB$ , $c=gC$ and $l=bx_0+cy_0=g(Bx_0+Cy_0)$. Thus $g|l $ and so by Theorem 1.1 (5) we conclude that $g \leq l$. We know $g<l$ is impossible since $g$ is the greatest common divisor, so $g=l=bx_0+cy_0$.

% Lecture 3

\textbf{Theorem 1.4:}\\
The greatest common denominator of $b$ and $c$ can be characterised in the following two ways:
\begin{enumerate}
    \item It is the least positive value of $bx+cy$ where $x,\ y\in\mathbb{Z}$
    \item If $d$ is any common divisor of $b$ and $c$ then $d|g$ by Theorem 1.1 (3).
\end{enumerate}

\textbf{Proof:} Theorem 1.4

\begin{enumerate}
\item Follows from Theorem 1.3
\item If $d$ is any common divisor of $b$ and $c$, then $d|g$ by Theorem 1.1 (3). Moreover, there cannot be two distinct integers with property (2), because of Theorem 1.1 (4).
\end{enumerate}

Note: If $d=bx+cy$ , then $d$ is not necessary the $gcd(b,c)$. However, it does follow from such equation that $(b,c)$ is a divisor of $d$. In particular , if $bx+cy=1$ for some $x,y\in\mathbb{Z}$ , then $(b,c)=1$.

\textbf{Theorem 1.5:}\\
Given $b_1,...,b_n\in\mathbb{Z}$ not all zero with greatest common divisor $g$, there exists integers $x_1,...,x_n$,  such that
\begin{align}
    g=(b_1,...,b_n)=\sum^n_{j=1}b_jx_j
\end{align}
Furthermore, g is the least positive value of the linear form $\sum^n_{j=i}b_jy_j$ where the $y_j$ runs over all integers; also $g$ is the positive common divisor of $b_1,...,b_n$ that is divisible by every common divisor.

\textbf{Proof:} Theorem 1.5

Exercise for the reader.



\textbf{Theorem 1.6:}\\
For any $m\in\mathbb{Z}, m>0$
\begin{align}
    (ma,mb)=m(a,b)
\end{align}

\textbf{Proof:} Theorem 1.6

By Theorem 1.4 we have:

$(ma,mb)$ = least positive value of $max+mby$ =$m$ \{ least positive integer of $ax+by$\} = $m(a,b)$

\textbf{Theorem 1.7:}\\
If $d|a$, $d|b$ and $d>0$, then
\begin{align}
    \bigg(\frac{a}{d},\frac{b}{d}\bigg)=\frac{1}{d}(a,b)
\end{align}
If $(a,b)=g$, then
\begin{align}
    \bigg(\frac{a}{g},\frac{b}{g}\bigg)=1
\end{align}



\textbf{Proof:} Theorem 1.7

The second assertion is the special case of the first using $d=(a,b)=g$. The first assertion is a direct consequence of Theorem 1.6, obtained by replacing $m,a,b$ in Theorem 1.6 by $d,\frac{a}{d},\frac{b}{d}$ respectively.

\textbf{Theorem 1.8:}\\
If $(a,m)=(b,m)=1$ then $(ab,m)=1$

\textbf{Proof:} Theorem 1.8

Exercise for the reader.

\textbf{Definition:} 1.3

We say that $a$ and $b$ are \underline{relatively prime} in case $(a,b)=1$, and that $a_1,a_2,...,a_n$ are relatively prime in the case $(a_1,a_2,...,a_n)=1$. We say that $a_1,a_2,...,a_n$ are \underline{relatively prime in pairs} in case $(a_i,a_j)=1$ for all $i=1,2,...,n$ and $j=1,2,...n$ with $i \neq j$.

Note: $(a,b)=1$ we also say $a$ and $b$ are \underline{coprime}.

\textbf{Theorem 1.9:}\\
For any $x\in\mathbb{Z}$ we have
\begin{align}
(a,b)=(b,a)=(a,-b)=(a,b+ax)
\end{align}

\textbf{Proof:} Theorem 1.9

Exercise for the reader.

\textbf{Theorem 1.10:} Euclid's Lemma\\
If $c|ab$ and $(b,c)=1$, then $c|a$.

\textbf{Proof:} Theorem 1.10

By Theorem 1.6 , $(ab,ac)=a(b,c)=a$. By hypothesis $c|ab$ and clearly $c|ac$, so $c|a$ by Theorem 1.4 (2).

Now we observe for $c\neq 0$ , we have $(b,c)=(b,-c)$ by Theorem 1.9 and hence we may presume $c>0$.

\textbf{Theorem 1.11:} The Euclidean Algorithm\\
Given $b,c\in\mathbb{Z}, c>0$, we can make a repeated application of the division algorithm, \textbf{Theorem 1.2}, to obtain a series of aligns
\begin{align}
b=cq_1+r_1          & \quad0<r_1<c\\
c=r_1q_2+r_2        & \quad0<r_2<r_1\\
r_1=r_2q_3+r_3      & \quad0<r_3<r_2\\
...\\
r_j=r_{j+1}q_j+r_j  & \quad0<r_j<r_{j-1}\\
r_{j-1}=r_jq_{j+1}.
\end{align}
The greatest common divisor $(b,c)$ of $b$ and $c$ is $r_j$, the last nonzero remainder in the division process. Values of $x_0$ and $y_0$ in $(b,c)=bx_0+cy_0$ can be obtained by writing each $r_i$ as a linear combination of $b$ and $c$.

\textbf{Proof:} Theorem 1.11

See Theorem 1.11 in the textbook or Theorem 1.13 in the Lecture Notes.

% Lecture 4

\textbf{Example 1}
$gcd(841,160)$
\begin{align}
\begin{split}
841&=160\times5 + 41 \\
160&=41\times3 + 37 \\
41&=37\times 1 + 4 \\
37&=34\times 9 + 1 \\
4&=1\times 4 + 0
\end{split}
\end{align}

Hence (841,160)=1 working backwards gives:


\begin{align}
1&=37\times1 - 4\times9 \\
1&=37\times1 - (41-37)\times9 \\
1&=37\times10 - 41\times9 \\
1&=(160-3\times41)\times10 - 41 \times 9 \\
1&=160\times10 - 41\times39 \\
1&=160\times10 - (841-160\times5)\times39 \\
1&=(-39)\times841 + 205\times160
\end{align}

Note the solution is not unique:
\begin{align}
1=121\times841 - 636\times 160
\end{align}

\textbf{Example 2} Extended Algorithm


\begin{align}
\begin{split}
r_i&=r_{i-2} - q_ir_{i-1} \\
x_i&=x_{i-2} - q_ix_{i-1} \\
y_i&=y_{i-2} - q_iy_{i-1} \\
r_1&=b , r_0=c \\
x_1&=1 , x_0=0 \\
y_1&=0 , y_0=1
\end{split}
\end{align}

We want to compute the $gcd(841,160)$ and express as a linear combination of 841 and 160.

\textbf{Definition 1.4:}\\
The integers $a_1,...,a_n$, all different from zero, have a \textbf{common multiple} $b$ if $a_i|b$ for $i=1,...,n$. The least of the positive common multiples is called the \textbf{least common multiple} and it is denoted by $[a_1,...,a_n]$ or $lcm(a1,...,a_n)$

\textbf{Theorem 1.12:}\\
If $b$ is any common multiple of $a_1,...,a_n$, then $[a_1,...,a_n]\ |\ b$. This is the same as saying that if $h=[a_1,...,a_n]$ then $0,\pm h,\pm 2h,...$ comprise all the common multiples of $a_1,...,a_n$.

\textbf{Proof:} Theorem 1.12

Let $m$ be any common multiple and divide $m$ and $h$. By Theorem 1.2 , $\exists q,r$ such that $m=qh+r$ , $ 0\leq r<h.$ We must probe that $r=0$. If $r\neq0$ we argue as follows. For each $i=1,2,...,n$ we know that $a_i|h$ and $a_i|m$, so that $a_i|r$ . Thus $r$ is a positive common multiple of $a_1,a_2,...,a_n$ contrary to the fact that h is the least of all positive common multiples.


\textbf{Theorem 1.13:}\\
If $m>0$
\begin{enumerate}
    \item $[ma,mb] = m[a,b]$
    \item $[a,b](a,b)=|ab|$
\end{enumerate}

\textbf{Proof:} Theorem 1.13

\begin{enumerate}
\item Let $H=[ma,mb]$ and $h=[a,b]$. Then $mh$ is a multiple of $ma$ and $mb$, so that $mh\geq H$. Also, $H$ is a multiple of both $ma$ and $mb$ so $H/m$ is a multiple of $a$ and $b$. Thus, $H/m \geq h$ from which it allows that $mh=H$.
\item It will suffice to prove this for $a,b\in\mathbb{Z}$ with $a>0$, $b>0$, since $[a,-b]=[a,b]$.\\v
We begin with the special case where $(a,b)=1$. Now $[a,b]$ is a multiple of a, say $ma$. Then $b|ma$ and $(a,b)=1$, so by Theorem 1.10 we conclude that $b|m$. Hence $b \leq m$ , $ba\leq ma$. But $ba$, being a positive common multiple of $b$ and $a$ , cannot be less than the least common multiple, so $ba=ma=[a,b]$.

Let $(a,b)=g>1$. we have $(a/g,b/g)=1$ by Theorem 1.7. Applying the result of the previous paragraph we have:
\begin{align}
\bigg[\frac{a}{g},\frac{b}{g}\bigg]\cdot\bigg(\frac{a}{g},\frac{b}{g}\bigg)=\frac{a}{g}\frac{b}{g}
\end{align}

Multiplying by $g^2$ and using Theorem 1.6 as well as the first part (1.), we get $[a,b]\cdot(a,b)=ab$.
\end{enumerate}

% Up to where Chris finished proofs and Examples

\subsection{Primes}

\textbf{Definition 1.5:}\\
An integer $p>1$ is called a \textbf{prime number} if there is no divisor $d$ of $p$ satisfying $1<d<p$. If an integer $a>1$ is not a prime, is is called a \textbf{composite number}.

\textbf{Theorem 1.14:}\\
Every integer $n>1$ can be expressed as a product of primes (with perhaps only one factor).

\textbf{Proof:} Theorem 1.14\\
If $n=p$ a prime then it is a 'product' with one factor.
Otherwise $n=n_1n_2$, where $1<n_1,n$ and $1<n_2,n$. If $n_1$ is a prime, let it stand; otherwise it will factor into, $n_1=n_3n_4$, with $1<n_3,n$ and $1<n_4,n$, similarly for $n_2$. This process of writing each composite number that arises as a product of factors must terminate because the factors are smaller then the composite number itself, and yet each factor is an integer greater then $1$. Thus in the end we have
\begin{align*}
    n = p_1^{a_1}p_2^{a_2}...p_r^{a_r}
\end{align*}
since the prime factors are not necessarily distinct where $p_1,p_2,...,p_r$ are distinct primes and $a_1,a_2,...,a_r>0$

% Lecture 5

\textbf{Theorem 1.15:}\\
If $p|ab$, p prime, then $p|a$ or $p|b$.\\
More generally if $p|a_1...a_n$, then $p$ divides at least on of the factors $a_i$

\textbf{Proof:} Theorem 1.15\\
If $p\nmid a$, then $(a,p)=1$ and so by \textbf{Thm 1,10}, $p|b$. For the general case, we use induction.

\textbf{Theorem 1.16:} Fundamental Theorem of Arithmatic\\
The factoring of any integer $n>1$ into primes is unique apart from the order of the prime factors.

\textbf{Definition 1.6:}\\
We call $a$ a square (or \textbf{perfect square}) if it can be written as $a=n^2$. By the \textbf{F.T.A.} $a$ is a square if all the exponents $\alpha(p)$ in (1.6) are even. We say that $a$ is \textbf{square free} if 1 is the largest square dividing $a$. Thus $a$ is square free iff the exponents $\alpha(p)=0$ or $1$. If p is prime, then the assertion $p^k||a$ is equivalent to $k=\alpha(p)$.

\textbf{Theorem 1.17:} (Euclid)\\
The number of primes is infinite.

\textbf{Definition 1.7:}\\
Let $n\in\mathbb{N}$ and $p$ a prime. Then
\begin{align}
    v_p(n) = max(k\in\mathbb{N}_{\cup0}:p^k|n)
\end{align}
where k is the unique non-negative integer such that $p^k|n$ but $p^{k+1}|n$\\
Equivalently $V_p(n)=k$ iff $n=p^kn'$ where $n'\in\mathbb{N}$ and $p\nmid n'$

\textbf{Lemma:}
Let $n,m\in\mathbb{N}$ and $p$ be a prime. then
\begin{align}
    v_p(mn)=v_p(m)+v_p(n)
\end{align}

% Lecture 6

\section{Congruences}

\subsection{Congruences}

\textbf{Definition 2.1:}\\
If $m\in\mathbb{Z}$, $m\neq0$ is such that $m|a-b$, we say that $a$ is \underline{congruent to} $b$ modulo $m$ and we write $a\equiv b\ (mod\ m)$

Since $a-b$ is divisible by $-m$, we can focus our attention to a positive modulus. We will assume in this chapter that $m>0$.

\textbf{Theorem 2.1:} Properties of Congruences
\begin{enumerate}
    \item $a\equiv b\ (mod\ m)$ $b\equiv a\ (mod\ m)$, and $a-b\equiv 0\ (mod\ m)$ are equivalent statements.
    \item If $a\equiv b\ (mod\ m)$ and $b\equiv c\ (mod\ m)$, then $a\equiv c\ (mod\ m)$
    \item If $a\equiv b\ (mod\ m)$ and $c\equiv d\ (mod\ m)$, then $a+c\equiv b+d\ (mod\ m)$
    \item If $a\equiv b\ (mod\ m)$ and $c\equiv d\ (mod\ m)$, then $ac\equiv bd\ (mod\ m)$
    \item If $a\equiv b\ (mod\ m)$ and $d|m,\ d>0$, then $a\equiv b\ (mod\ d)$
    \item If $a\equiv b\ (mod\ m)$ then $ac\equiv bc\ (mod\ mc)$ for $c>0$
\end{enumerate}

\textbf{Theorem 2.2:}\\
Let $f$ denote a polynomial with integral coefficients. If $a\equiv b\ (mod\ m)$ then $f(a)\equiv f(b)\ (mod\ m)$

\textbf{Theorem 2.3:}
\begin{enumerate}
    \item If $ax\equiv ay\ (mod\ m)$ iff $x\equiv y\ (mod\ \frac{m}{(a,m)})$
    \item $ax\equiv ay\ (mod\ m)$ and $(a,m)=1$, then $x\equiv y\ (mod\ m)$
    \item $x\equiv y\ (mod\ m_i)$ for $i=1,...,r$ iff $x\equiv y\ (mod\ [m_1,...,m_r])$
\end{enumerate}

% Lecture 7

\textbf{Theorem 2.4:}\\
If $b\equiv c\ (mod\ m)$, then $(b,m)=(c,m)$.

\textbf{Proof:} Theorem 2.4\\
We have $b=c+mx$ for some integer $x$. To see that $(b,m)=(c+ax,m)$, take $a=m$ in Theorem 1.9.

\textbf{Definition 2.2:} Complete Residue System\\
If $x\equiv y\ (mod\ m)$ then y is called a \underline{residue} of $x\ (mod\ m)$. A set $y_1,...,y_m$ is called a \underline{complete residue system} modulo $m$ if for every integer $x$, there is one and only one $y_j$ such that $x \equiv y_j\ (mod\ m)$

\textbf{Definition 2.3:} Reduced Residue System\\
A \underline{reduced residue system} modulo $m$ is a set of integers $r_i$ such that\\
$(r_i,m)=1$,\\
$\ r_i\not\equiv r_j,\ (mod\ m)$ if $i\neq j$,\\
for every $x$ where $(x,m)=1$, there is $r_i$ in the set such that $x \equiv r_i \ (mod\ m)$.

\begin{itemize}
    \item You can obtain a reduced residue system by deleting from a complete residue system modulo $m$ those members that are not relatively prime to $m$.
    \item All reduced reside system modulo $m$ have the same number of elements.
    \item We will denote by $\Phi(m)$ to be the number of elements of a reduced residue system modulo $m$.
    \item $\Phi(m)$ is called the \underline{Euler's $\Phi$-function} or \underline{Euler's totient-function}
\end{itemize}

\textbf{Theorem 2.5:}\\
The number $\Phi(m)$ is the number of positive integers less than or equal to $m$ that are relatively prime to $m$.

\textbf{Theorem 2.6:}\\
Let $(a,m)=1$. Let $r_1,...,r_n$ be a complete, or a reduced, residue system modulo $m$. Then $ar_1,...,ar_n$ is a complete, or a reduced, residue system, respectively, modulo $m$.

\textbf{Proof:} Theorem 2.6\\
If $(r_i,m)=1$, then $(ar_i,m)=1$ by Theorem 1.8.\\
There are the same number of $ar_1,ar_2,...,ar_n$ as of $r_1,r_2,...,r_n$. Therefore we only need to show that $ar_i \not \equiv  ar_j \ (mod\ m)$ if $i\neq j$. But Theorem 2.3 (2) shows that $ar_i \equiv ar_j \ (mod\ m)$ implies $r_i \equiv r_j \ (mod\ m)$ and hence $i=j$.

% Lecture 8

\textbf{Theorem 2.7:} Fermat's Theorem\\
Let $p$ denote a prime. If $p\nmid a$ for $a\in\mathbb{Z}$ then\\
$a^{p-1}\equiv 1\ (mod\ p)$.\\
$a^{p}\quad\equiv a\ (mod\ p)$.

\textbf{Theorem 2.8:} Euler's Generalization of Fermat's Theorem\\
If $(a,m)=1$, then
\begin{align}
    a^{\Phi(m)}\equiv 1\ (mod\ m)
\end{align}

\textbf{Proof:} Theorem 2.8\\
Let $r_i,r_2,...,r_{\Phi(m)}$ be a reduced residue system $(mod\ m)$. Then by Theorem 2.6, $ar_1,ar_2,...,ar_{\Phi(m)}$ is also a reduced residue system $\ (mod\ m)$.  Hence, corresponding to each $r_i$ there is only one $ar_j$ such that $r_i \equiv ar_j \ (mod\ m)$. Furthermore, different $r_i$ will have different corresponding $ar_j$. This means that the numbers $ar_1,ar_2,...,ar_{\Phi(m)}$ are just the residues $(mod\ m)$ of $r_1,r_2,...,r_{\Phi(m)}$, but not necessarily in the same order. Multiplying and using Theorem 2.1 (4) we obtain
\begin{align*}
                \prod_{j=1}^{\Phi(m)}(ar_j) &\equiv  \prod_{i=1}^{\Phi(m)} r_i \ (mod\ m)\\
    a^{\Phi(m)} \prod_{j=1}^{\Phi(m)}(r_j)\ &\equiv  \prod_{i=1}^{\Phi(m)} r_i \ (mod\ m)
\end{align*}
Now $(r_j,m)=1$, so we can use Theorem 2.3 (2), to cancel the $r_j$ and we obtain $a^{\Phi(m)} \equiv 1 \ (mod\ m)$.

\textbf{Proof:} Theorem 2.7\\
To find $\Phi(p)$, we refer to Theorem 2.5. All the integers $1,2,...,p-1,p$ with the exception of p are relatively prime to $p$. This we have $\Phi(p)=p-1$ and the first part of Fermat's Theorem follows.\\
The second part is obtained from the first, multiplying by $a$.

\textbf{Theorem 2.9:}\\
If $(a,m)=1$ then there is an $x$ such that $ax\equiv 1\ (mod\ m)$. Any two such $x$ are congruent $(mod\ m)$. If $(a,m)>1$ then there is no such $x$.

\textbf{Proof:} Theorem 2.9\\
If $(a,m)=1$, then there exists $x,y\in\mathbb{Z}$ such that $ax+my=1$. That is, $ax \equiv 1 \ (mod\ m)$. Conversely, if $ax \equiv 1 \ (mod\ m)$, then there is a $y$ such that $ax+my=1$, that that $(a,m)=1$. Thus if $ax_1 \equiv ax_2 \equiv 1 \ (mod\ m)$ then $(a,m)=1$ and it follows from Theorem 2.3 (2) that $x_1 \equiv x_2 \ (mod\ m)$

\textbf{Lemma 2.10:}\\
Let p be a prime number. Then $x^2\equiv 1\ (mod\ p)$ iff $x\equiv\pm 1\ (mod\ p)$.

\textbf{Theorem 2.11:} Wilson's Theorem\\
If $p$ is prime, then $(p-1)!\equiv -1\ (mod\ p)$

\textbf{Theorem 2.12:}\\
Let $p$ denote a prime. Then $x^2\equiv -1\ (mod\ p)$ has solutions iff $p=2$ or $p\equiv 1\ (mod\ 4)$.

% Lecture 9

\textbf{Proof:} Theorem 2.12\\

\textbf{Lemma 2.13:}\\
If $p$ is prime and $p\equiv 1\ (mod\ 4)$, then there exists positive integers
$a$ and $b$ such that $a^2+b^2=p$.

% Lecture 10

\textbf{Lemma 2.14:}\\
Let $q$ be a prime factor of $a^2+b^2$. If $q\equiv 3\ (mod\ 4)$ then $q|a$ and $q|b$.


\textbf{Theorem 2.15:} (Fermat)\\
Let
\begin{align}
    n =  2^\alpha\prod_{p\equiv 1(4)}p^\beta\prod_{q\equiv 3(4)}q^\gamma
\end{align}
Then $n$ can be expressed as a sum of two squares iff all the exponents of $\gamma$ are even.

\subsection{Solutions of Congruences}
\begin{itemize}
    \item Let $f(x)$ denote a polynomial, e.g.
    \begin{align}
        f(x)=a_nx^n+...+a_0
    \end{align}
    \item if $u\in\mathbb{Z}$ such that $f(u)\equiv0\ (mod\ m)$ then we say that $u$ is a\\ \underline{solution of the congruence} $f(x)\equiv0\ (mod\ m)$
    \item If $u$ is a solution of $f(x)\equiv0\ (mod\ m)$ and if $v\equiv u\ (mod\ m)$, then Theorem 2.2 shows that $v$ is also a solution.
    \begin{itemize}
        \item $x\equiv u\ (mod\ m)$ is a solution of $f(x)\equiv0\ (mod\ m)$ meaning that every integer congruent to $u$ modulo $m$ satisfied $f(x)\equiv0\ (mod\ m)$.
    \end{itemize}
\end{itemize}

\textbf{Definition 2.4:}\\
Let $r_1,...,r_m$ denote a complete residue system modulo $m$.\\
The \underline{number of solutions} of $f(x)\equiv0\ (mod\ m)$ is the number of the $r_i$ such that $f(r_i)\equiv0\ (mod\ m)$

% Lecture 11

\textbf{Definition 2.5:}\\
Let $f(x)=a_nx^n+...+a_0$. If $a_n\not\equiv0\ (mod\ m)$ the \underline{degree of the congruence} $f(x)\equiv0\ (mod\ m)$ is $n$. If $a_n\equiv0\ (mod\ m)$, let $j$ be the largest integer such that $a_j\not\equiv0\ (mod\ m)$; then the degree of the congruence is $j$. If there is no such integer $j$, then no degree is assigned to the congruence.

\textbf{Theorem 2.16:}\\
If $d|m$, $d>0$, and if $u$ is a solution of $f(x)\equiv0\ (mod\ m)$, then $u$ is a solution of $f(x)\equiv0\ (mod\ d)$
\begin{itemize}
    \item We say that $f(x)\equiv0\ (mod\ m)$ is an \underline{identical congruence} if it holds for all integers $x$
    \begin{itemize}
        \item If $f(x)$ is a polynomial whose coefficients are divisible by $m$, then $f(x)\equiv0\ (mod\ m)$ is an identical congruence
        \item $x^p\equiv x\ (mod\ p)$ is true for all integers $x$ by Fermat's \textbf{Theorem :}\\

    \end{itemize}
\end{itemize}

\textbf{Theorem 2.17:} Linear Congruences\\
Let $a,b$ and $m>0$ be given integers, and put $g=(a,m)$. The congruence $ax\equiv b\ (mod\ m)$ has a solution iff $g|b$. If this condition is met, then the solution forms an arithmetic progression with common difference $\frac{m}{g}$, giving $g$ solutions $(mod\ m)$.

\textbf{How to solve general linear congruences:}
Let $a,b\in\mathbb{Z}$ and let $n\in\mathbb{N}$. Suppose we wish to solve the linear congruence
\begin{align}
    ax\equiv b\ (mod\ n)
\end{align}
Firstly apply the Extended Euclidean Algorithm to compute $d=gcd(a,n)$ and find $x_0,y_0\in\mathbb{Z}$ such that
\begin{align}
    ax_0+ny_0=d
\end{align}
If $d\nmid b$ then there are no solutions by Theorem 2.17. Otherwise, there are exactly $d$ solutions modulo $n$ by Theorem 2.17, which we can find as follows\\
Write
\begin{align}
    a=da',\quad b=db',\quad n=dn'
\end{align}
Dividing by $d$ gives
\begin{align}
    a'x_0+n'y_0=1
\end{align}
Then reducing mod $n'$ gives
\begin{align}
    a'x_0\equiv1\ (mod\ n')
\end{align}
and multiplying by $b'$ gives
\begin{align*}
    a'(b'x_0)\equiv b'\ (mod\ n')
\end{align*}
Therefore $t=b'x_0$ is the unique solution to $a'x\equiv b'\ (mod\ n')$. Now by Theorem 2.17 the solutions to (17) are $t,t+n',...,t+(d-1)n'$

% Lecture 12

\subsection{The Chinese Remainder Theorem}
Solve Simultaneous Congruences

Find x (is there are any) that satisfies
\begin{align}
    \begin{split}
        x & \equiv a_1\ (mod\ m_1)\\
        x & \equiv a_2\ (mod\ m_2)\\
          & ...\\
        x & \equiv a_r\ (mod\ m_r)\\
    \end{split}
\end{align}

\textbf{Theorem 2.18:} The Chinese Remainder Theorem\\
Let $m_1,...,m_r$ denote $r$ positive integers that are relatively prime in pairs, and let $a_1,...,a_r\in\mathbb{Z}$. Then the congruences (21) have have common solutions. If $x_0$ is one such solution, then an integer $x$ satisfies the congruences (21) iff $x=x_0+km$ for some integer $k$. Here $m=m_1m_2...m_r$

\textbf{Proof:} Theorem 2.18\\
Let $m = m1m2...mr$, we see that $\frac{m}{m_j}\in\mathcc{Z}$ and that $(\frac{m}{m_j},m_j)=1$. Hence by theorem 2.9, for each $j$ there is an integer $b_j$ such that $(\frac{m}{m_j},m_j)b_j \equiv 1 \ (mod\ m_j)$. Clearly $(\frac{m}{m_j},m_j)b_j \equiv 0 \ (mod\ m_i)$ if $i\not\equiv j$. Put
\begin{align*}
    x_0 = \sum_{j=1}^r\frac{m}{m_j}b_ja_j
\end{align*}
We consider this number modulo $m_i$, and find that
\begin{align*}
    x_0 \equiv \frac{m}{m_i}b_ia_i \equiv a_i \ (mod\ m_i)
\end{align*}
Thus $x_0$ is a solution of the system (21).
If $x_0$ and $x_1$ are two solutions of the system (21), then $x_0 \equiv x_1 \ (mod\ m_i)$ for $i=1,2,...,r$ and hence $x_0 \equiv x_1 \ (mod\ m)$ by theorem 2.3 (3). This completes the proof.

% Lecture 13

\begin{itemize}
    \item $m_1,...,m_r$ positive integers relatively prime in pairs
    \item $m=m_1m_2...m_r$
    \item Instead of considering just one set of aligns, we will consider all possible systems of this type
    \item Let
    \begin{align}
        \begin{split}
            a_1&\in\{1,...,m_1\}\\
            a_2&\in\{1,...,m_2\}\\
            &...\\
            a_r&\in\{1,...,m_r\}
        \end{split}
    \end{align}
    \item The number of such $r$-tuples $(a_1,...,a_r)$ is $m=m_1m_2...m_r$.
    \item By the \textbf{C.R.T.} each $r$-tuple determines precisely one residue class $x$ modulo $m$.
    \begin{itemize}
        \item Moreover, distinct $r$-tuples determine different residue classes. To see this, suppose that $(a_1,...,a_r)\neq(a_1',...,a_r')$. then $a_i\neq a_i'$ for some $i$, and we see that no integer $x$ satisfies both the congruences $x\equiv a_i\ (mod\ m_i)$ and $x\equiv a_i'\ (mod\ m_i)$
    \end{itemize}
    \item Thus we have a one-to-one correspondence between the $r$-tuples $(a_1,...,a_r)$ and a complete residue system modulo $m$, such as the integers $1,...,m$
\end{itemize}

\textbf{Theorem 2.19:}\\
If $m_1,\ m_2>0,\ (m_1,m_2)=1$, then $\Phi(m_1m_2)=\Phi(m_1)\Phi(m_2)$ moreover, if $m=\Pi p^\alpha$ then
\begin{align}
    \Phi(m)=\prod_{p|m}(p^\alpha-p^{\alpha-1})=m\prod_{p|m}(1-\frac{1}{p})
\end{align}

\textbf{Proof:} Theorem 2.19\\
If $m=1$, then the products are empty, and by convention an empty product has value $1$. Thus $\Phi(1)=1$ in this case, which is correct.

Put $m=m_1m_2$, and suppose $(x,m)=1$. By reducing $x$ modulo $m_1$, we see that there is a unique $a_1\in\mathcal{C}(m_1)$ for which $x \equiv a_1 \ (mod\ m_1)$. Similarly there is a unique $a_2\in\mathcal{C}(m_2)$ for which $x \equiv a_2 \ (mod\ m_2)$. Since $(x_1,m_1)=1$, it follows by theorem 2.4 that $(a_1,m_1)=1$. Similarly $(a_2,m_2)=1$. For any $n\in\mathcc{Z},\ n>0$, let $\mathcal{R}(n)$ be the system of reduced residues formed of those numbers $a\in\mathcal{C}(n)$ for which $(a,n)=1$. That is $\mathcal{R}(n) = \{a\in\mathcal{C}(n):(a,n)=1\}$. Thus we see that any $x\in\mathcal{R}(m)$ gives rise to a pair $(a_1,a_2)$ with $a_i\in\mathcal{R}(m_i)$ for $i=1,2$.
Suppose, conversely, that we start with such a pair. By the CRT there exists a unique $x\in\mathcal{C}(m)$ such that $x \equiv a_i \ (mod\ mi)$ for $i=1,2$. Since $(a_1,m_1)=1$ and $x \equiv a_1 \ (mod\ m_1)$, it follows by theorem 2.4 that $(x_1,m_1)=1$. Similarly we find that $(x,m_2)=1$ and hence $(x,m)=1$.
That is $x\in\mathcal{R}(m)$. In this way we see that CRT enables us to establish a one to one correspondence between the reduced residue classes modulo $m$ and pairs of reduced residue classes modulo $m_1$ and $m_2$, provided that $(m_1,m_2)=1$.
Since $a_1\in\mathcal{R}(m_1)$ can take any one of $\Phi(m_1)$ values and the same for $a_2$ taking $\Phi(m_2)$ values, there are $\Phi(m_1)\Phi(m_2)$ pairs, so that $\Phi(m)=\Phi(m_1)\Phi(m_2)$. If $m=\prod p^\alpha$, then by repeated use of the above identity, we see that $\Phi(m)=\prod\Phi(p^\alpha)$. If $a$ is one of the $p^\alpha$ numbers $1,2,...,p^\alpha$, then $(a,p^\alpha)=1$ unless $a$ is one of the $p^{\alpha-1}$ numbers, $p,2p,...,p^{\alpha-1}p$.
on subtracting, we deduce that the number of reduced residue classes modulo $p^\alpha$ is $p^\alpha-p^{\alpha-1}=p^\alpha(1-frac{1}{p})$. This gives the stated formula.

% Lecture 14

\textbf{Theorem 2.20:}\\
Let $f(x)$ be a fixed polynomial with integral coefficients, and for any positive integer $m$ let $N(m)$ denote the number of solutions of the congruence $f(x)\equiv0\ (mod\ m)$. If $m=m_1m_2$ where $(m_1,m_2)=1$, then $N(m)=N(m_1)N(m_2)$. If $m=\prod p^\alpha$, then $N(m)=\prod N(p^\alpha)$

\subsection{Public-key Cryptography}

\textbf{Lemma 2.22:}\\
Suppose $m\in\mathbb{Z},\ m>0,\ (a,m)=1$. If $k,\overline{k}\in\mathbb{Z}$ and $k,\overline{k}>0$ such that $k,\overline{k}\equiv1\ (mod\ \Phi(m))$, then $a^{k\overline{k}}\equiv a\ (mod\ m)$.

\textbf{Proof:} Theorem 2.22\\
Write $k\overline{k}=1+r\Phi(m)$ for some $r\in\mathbb{Z}$. Then by Euler's congruence
\begin{align*}
    a^{k\overline{k}} = aa^{r\Phi(m)} = a(a^{\Phi(m)})^r \equiv a\cdot1^r = a\ (mod\ m)
\end{align*}

\begin{itemize}
    \item If $(a,m)=1,\ k>0$, then $(a^k,m)=1$. Thus if $n=\Phi(m)$ and $r_1,...,r_n$ is a system of reduced residues $(mod\ m)$, then the numbers $r_1^k,...,r_n^k$ are also relatively prime to $m$. These $k^\text{th}$ powers may not all be distinct $(mod\ m)$, as we see by considering the case $k=\Phi(m)$. On the other hand, from lemma 2.22, we can deduce that these $k^\text{th}$ powers are distinct $(mod\ m)$ provided that $(k,\Phi(m))=1$.
    \item Suppose that $r_i^k\equiv r_j^k\ (mod\ m)$ and $(k,\Phi(m))=1$. By Theorem 2.9 we may find $\overline{k}>0$ such that $k\overline{k}\equiv 1\ (mod\ \Phi{m})$ and then it follows from the lemma that
    \begin{align}
        r_i\equiv r_i^{k\overline{k}}=(r_i^k)^{\overline{k}}\equiv (r_j^k)^{\overline{k}}= r_j^{k\overline{k}}\equiv r_j\ (mod\ m)
    \end{align}
    This implies that $i=j$. We will show later that the converse also holds: the numbers $r_i^k,...,r_n^k$ are distinct $(mod\ m)$ only if $(k,\Phi(m))=1$. Suppose that $(k,\Phi(m))=1$. Since the numbers $r_1,...,r_n^k$ are distinct $(mod\ m)$, they form a system of reduced residues $(mod\ m)$. That is the map $a\mapsto a^k$ permutates the reduced residues $(mos\ m)$ if $(k,\Phi(m))=1$. The significance of the lemma is that the further map $b\mapsto b^{\overline{k}}$ is the inverse permutation.
    \item To apply these observations to cryptography, we take two distinct large primes, $p_1, p_2$, say each one with about 100 digits.
    \begin{itemize}
        \item So $m=p_1p_2$ has about 200 digits.
        \item Since we know the prime factorisation of m, from Theorem 2.19 we have that $\Phi(m)=(p_1-1)(p_2-1)$
        \item So $\Phi(m)<m$
        \item we choose now a big number $k$, $0<k,\Phi(m)$ and check by the Euclidean algorithm that $(k,\Phi(m))=1$. We try until we get such a $k$.
        \item We make the numbers $m$ and $k$ publicly available, by keep $p_1,p_2$ and $\Phi(m)$ secret.
        \item suppose now thatt some associate of ours wants to send us a message, say \textit{'Gauss was a genuis!'}. The associate first converts the characters to number in some standard way, say by emplying (ASCII). Then $G=071$, $a=097$,..., $!=033$. Then concatenate these codes to form a number
        \begin{align*}
            a=071097117115115126119097115126097126103101110105117115033
        \end{align*}
        \item if the message were longer, it could be ficided into a number of blocks.
        \item the associate could send the number $a$ and we could reconstruct the message. But suppose that message has some sensitive information. In that case the associate would use the number $k$ and $m$ that we have provided.
        \item Our associate quickly finds the unique number $b$, $0\leq b<m$ such that $b\equiv a^k\ (mod\ m)$ and sends this $b$ to us.
        \item We use Euclidean Algorithm to find $\overline{k}>0$ such that\\
        $k\overline{k}\equiv 1\ (mod\ \Phi(m))$ and then we find the unique $c$ such that $0\leq c<m$, $c\equiv b^{\overline{k}}\ (mod\ m)$. From lemma 2.22 we deduce that $a=c$.
    \end{itemize}
    \item In theory it might happen that $(a,m)>1$ in which case the lemma does not apply, but the chances of this is $\approx\frac{1}{p_i}\approx 10^{-100}$. Suppose that some third party gain access to the numbers $m$, $k$ and $b$, and seeks to recover the number a. In principle, all that needs to be done is to factor $m$, which yields $\Phi(m)$, and hence $\overline{k}$. The problem of locating the factors of $m$ for a big number is not easy.
\end{itemize}

% Lecture 15

\subsection{Prime Power Moduli}

Let $f(x)$ be a polynomial with integer coefficients. Let $N(m)$ denote the number of solutions of $f(x)\equiv 0\ (mod\ m)$.

Suppose that $m=m_1m_2$, where $(m_1,m_2)=$. With a "little work", Theorem 2.19 shows that the roots of the congruence $f(x)\equiv 0\ (mod\ m)$ are in one-to-one correspondence with pairs $(a_1,a_2)$ in which $a_1$ runs over all roots of the congruence $f(x)\equiv 0\ (mod\ m_1)$ and \\
$a_2$ runs over all roots of the congruence $f(x)\equiv 0\ (mod\ m_2)$.

\begin{itemize}
    \item From Theorem 2.16 and Theorem 2.20 we have that the congruence $f(x)\equiv 0\ (mod\ m)$ has solutions iff it has solutions $(mod\ p^\alpha)$ for each prime power $p^\alpha$ exactly dividing $m$.
\end{itemize}

\textbf{Example:}
Let $f(x)=x^2+x+7$. Find all roots of $f(x)\equiv 0\ (mod\ 189)$, given that $189=3^3\cdot7$, that all roots $(mod\ 27)$ are $4$, $13$, and $22$, and that the roots $(mod\ 7)$ are $0$ and $6$.

\textbf{Solution:}
By the Eucliean algorithm and (2.2), we find that $x\equiv a_1\ (mod\ 27)$ and that $x\equiv a_2\ (mod\ 7)$ iff $x\equiv 28a_1-27a_2\ (mod\ 189)$. We let $a_1=4, 13, 22$ and $a_2=0, 6$. Thus we obtain the six solutions $13, 49, 76, 112, 139, 175\ (mod\ 189)$
\begin{itemize}
    \item The problem of solving a congruence is now reduced to the case of a prime-power modulus.
    \item To solve $f(x)\equiv 0\ (mod\ p^k)$ we start with solutions to $f(x)\equiv 0\ (mod\ p)$ and then move to $p^2, p^3,...,p^k$.
\end{itemize}

Suppose that $x=a$ is a solution of $f(x)\equiv 0\ (mod\ p^j)$ and we want to use it to get a solution modulo $p^{j+1}$. The idea is to try to get a solution $x=a+tp^j$, where t is to be determined, by use of Taylor's expansion
\begin{align}
    f(a+tp^j)=f(a)+tp^jf'(a)+t^2p^{2j}\frac{f''(a)}{2!}+...+t^np^{nj}\frac{f^{(n)}(a)}{n!}
\end{align}
where $n=$ degree of $f(x)$. All derivatives beyond the $n^\text{th}$ are identicallly zero. Now with respect to the modulus $p^{j+1}$, equation (37) gives
\begin{align}
    f(a+tp^j)\equiv f(a)+tp^jf'(a)\ (mod\ p^{j+1})
\end{align}
as the following argument shows. What we want to establish is that the coefficients of $t^1, t^3,..., t^n$ in (37) are divisible by $p^{j+1}$ and so can be ommited in (38). This is almost obvious because the powers of $p$ in those terms. The explanation is that $\frac{f^{(k)}(a)}{k!}$ is an integer for each value of $k$, $2\leq k\leq n$. To see this, let  $cx^r$ be a representative term from $f(x)$. The corresponding term in $f^{(k)}(a)$ is $cr(r-1)(r-2)...(r-k+1)a^{r-k}$.

% Lecture 16

We now use the fact (without proof), that the product of $k$ consecutive integers is divisible by $k!$, and the argument is complete. Thus, we have proved that the coefficients of $t^2, t^3,..., t^n$ in (37) are divisible by $p^{j+1}$. The congruence (38) reveals how $t$ should be chosen if $x=a+tp^j$ is to be a solution of $f(x)\equiv 0\ (mod\ p^{j+1})$. We want $t$ to be a solution of
\begin{align}
    f(a)+tp^jf'(a)\equiv 0\ (mod\ p^{j+1})
\end{align}
Since $f(x)\equiv 0\ (mod\ p^j)$ have the solutions $x=a$, we see that $p^j$ can be removed as a factor to give
\begin{align}
    tf'(a)\equiv -\frac{f(a)}{p^j}\ (mod\ p)
\end{align}
Which is a linear congruence in $t$. This congruence may have no solution, one solutions, or $p$ solutions. If $f'(a)\equiv 0\ (mod\ p)$, then this congruence has exactly one solution, and we obtain
\end{itemize}

\textbf{Theorem 2.3:} Hansel's Lemma:\\
Suppose that $f(x)$ is a polynomial with integral coefficients. If $f(a)\equiv0\ (mod\ p^j)$ and $f'(a)\not\equiv0\ (mod\ p)$ then there is a unique $t\ (mod\ p)$ such that $f(a+tp^j)\equiv0\ (mod\ p^{j+1})$

\begin{itemize}
    \item If $f(a)\equiv0\ (mod\ p^j)$, $f(b)\equiv 0\ (mod\ p^k)$, $j<k$ and $a\equiv b\ (mod\ p^j)$, then we say that \underline{$b$ lies above $a$}, or \underline{$a$ lifts to $b$}.
    \item If $a\equiv 0\ (mod\ p^j)$, then $a$ is called a \underline{non-singular} root if $f'(a)\not\equiv 0\ (mod\ p)$; otherwise it is \underline{singular}.
    \item By Hensel's lemma we see that a non-singular root $a\ (mod\ p)$ lifts to a unique root $a_2\ (mod\ p^2)$. Since $a_2\equiv a\ (mod\ p)$ it follows by Theorem 2.2 that $f'(a_2)\equiv f'(a)\not\equiv 0\ (mod\ p)$.\\
    By a second application of Hensel's lemma we may lift $a_2$ to form a root $a_3$ of $f(x)$ modulo $p^3$, and so on.
    \item In general we find that a non-singular root $a$ modulo $p$ lifts to a unique root $a_j$ modulo $p^j$ for $j=2,3,...$ by (2.5) we see that this sequence is generated by means of the recursion
    \begin{align}
        a_{j+1} = a_j-f(a_j)\overline{f'(a)}
    \end{align}
    where $\overline{f'(a)}$ is an integer chosen so that $f'(a)\overline{f'(a)}\equiv 1\ (mod\ p)$.
\end{itemize}

\textbf{Example:}
Solve $x^2+x+47\equiv 0\ (mod\ 7^3)$

\textbf{Solution:}
First we note that $x\equiv 1\ (mod\ 7)$ and $x\equiv 5\ (mod\ 7)$ are the only solutions of $x^2+x+47\equiv 0\ (mod\ 7)$. Since $f'(x)=2x+1$, we see that
\begin{itemize}
    \item $f'(1)=3\not\equiv 0\ (mod\ 7)$
    \item $f'(5)=11\not\equiv 0\ (mod\ 7)$
\end{itemize}
\textit{(So these roots are non singular)}

Taking $\overline{f'(1)}=5$, we see by (40) that the root $a\equiv 1\ (mod\ 7)$ lifts to $a_2=1-49\cdot5=-244$. Since $a_2$ is considered $(mod\ 7^2)$, we may take instead $a_2=1$. Then $a_3=1-49\cdot5\equiv 99\ (mod\ 7^3)$.\\
Similarly, we take $\overline{f'(5)}=2$ and see by (40) that the root $5\ (mod\ 7)$ lifts to $5-77\cdot2=-149\equiv 47\ (mod\ 7^2)$ and that $47\ (mod\ 7^2)$ lifts to $47-f(47)\cdot2=47-2303\cdot2=-4599\equiv 243\ (mod\ 7^3)$.\\
Thus we conclude that $99$ and $243$ are the desired roots and that there are no others.

% Lecture 17

\subsection{Prime Modulus}

$f(x)\equiv 0\ (mod\ m)$ $\to$ $f(x)\equiv 0\ (mod\ p)$\quad \textit{(reduced)}\\
(No general method exists to solve such congruences)

\textbf{Question:}\\
Given a polynomial congruence $f(x)\equiv 0\ (mod\ m)$ is there an analogue to the result in algebra which says that a polynomial equation of degree $n$ with complex coefficients has exactly $n$ roots?\\
$\to$ for congruences the solution is more complicated.

e.g. For any $m>1$, there are $f(x)$ such that $f(x)\equiv 0\ (mod\ m)$ has no solutions.

e.g.2 $x^p-x+1\equiv 0\ (mod\ m)$, where $p$ is a prime factor of $m$ has no solutions because $x^p-x+1\equiv 0\ (mod\ p)$ has none, by Fermat's Theorem.

$f(x)=a_nx^n+a_{n-1}x^{n-1}+...+a_1x+a_0$ and we assume $p\nmid a_n$ so that the congruence $f(x)\equiv 0\ (mod\ p)$ has degree $n$.

\textbf{Theorem 2.25:}\\
If the degree $n$ of $f(x)\equiv 0\ (mod\ p)$ is greater than or equal to $p$, then either every integer is a solution of $f(x)\equiv 0\ (mod\ p)$ or there is a polynomial $g(x)$ having integral coefficients, with leading coefficient $1$, such that $g(x)\equiv 0\ (mod\ p)$ is of degree less than $p$ and the solutions of $g(x)\equiv 0\ (mod\ p)$ are precisely those of $f(x)\equiv 0\ (mod\ p)$.

\textbf{Proof:} Theorem 2.25\\
Dividing $f(x)$ by $x^p-x$ we get a quotient $q(x)$ and a remainder $r(x)$ such that $f(x)=(x^p-x)q(x)+r(x)$. here $q(x)$ and $r(x)$ are polynomials with integral coefficients, and $r(x)=0$ or degree $r(x)<p$. Since every integer is a solutions of $x^p\equiv x\ (mod\ p)$ are the same as those of $r(x)\equiv 0\ (mod\ p)$ by Fermat's Theorem, we see that the solutions of $f(x)\equiv 0\ (mod\ p)$ are the same as those of $r(x)\equiv 0\ (mod\ p)$. If $r(x)=0$ or if every coefficient of $r(x)$ is divisible by $p$, then every integer is a solution of $f(x)\equiv 0\ (mod\ p)$.

On the other hand, if at least one coefficient of $r(x)$ is not divisible by $p$, then the congruence $r(x)\equiv 0\ (mod\ p)$ has a degree, and that degree is  less than $p$. The polynomial $g(x)$ in the Theorem can be obtained from $r(x)$ by getting leading coefficient $1$, as follows. We may discard all terms in $r(x)$ whose coefficients are divisible by $p$, since the congruence properties modulo $p$ are unaltered. Then let $bx^m$ be the term of the highest degree in $r(x)$, with $(b,p)=1$. Choose $\overline{b}$ so that $b\overline{b}\equiv 1\ (mod\ p)$, and note that $(\overline{b},b)=1$ also. Then the congruence $\overline{b}r(x)\equiv 0\ (mod\ p)$ has the same solutions as $r(x)\equiv 0\ (mod\ p)$, and so has the same solutions as $f(x)\equiv 0\ (mod\ p)$. Define $g(x)=\overline{b}r(x)$ with its leading coefficient $b\overline{b}$ replaced by 1, that is,
\begin{align}
    g(x)=\overline{b}r(x)-(b\overline{b}-1)x^m
\end{align}

\textbf{Theorem 2.26:}\\
The congruence $f(x)\equiv 0\ (mod\ p)$ of degree $n$ has at most $n$ solutions.

\textbf{Proof:} Theorem 2.26\\
The proof is by induction on the degree of $f(x)\equiv 0\ (mod\ p)$. If $n=0$, the polynomial $f(x)=a_0$ with $a_0\not\equiv 0\ (mod\ p)$ and hence the congruence has no solutions. If $n=1$, the congruence has exactly one solutions by Theorem 2.17. Assume the truth of the Theorem for all congruences of degree $<n$, supppose that there were more than $n$ solutions of the congruence $f(x)\equiv 0\ (mod\ p)$ of degree $n$. Let the leading term of $f(x)$ be $a_nx^n$ and let $u_1,...,u_{n+1}$ be solutions of the congruence with $u_1\not\equiv u_j\ (mod\ p)$ for $i\neq j$. We define $g(x)$ by
\begin{align}
    g(x)=f(x)-a_n(x-u_1)...(x-u_n)
\end{align}
noting the cancellation of $a_nx^n$ on the right.

Note that $g(x)\equiv 0\ (mod\ p)$ has at least $n$ solutions, namely $u_1,...,u_n$. We consider two cases:

\renewcommand{\theenumi}{\roman{enumi}}
\begin{enumerate}
    \item every coefficient. of $g(x)$ is divisible by $p$
    \item at least one coefficient is not divisible by $p$
\end{enumerate}

For (i), every integer is a solution of $g(x)\equiv 0\ (mod\ p)$, and since $f(u_{n+1})\equiv 0\ (mod\ p)$ by assumption, it follows that $x=u_{n+1}$ is a solutions of
\begin{align}
    a_n(x-u_1)...(x-u_n)\equiv 0\ (mod\ p)
\end{align}
This contradicts Theorem 1.15.

For (ii), we note that $g(x)\equiv 0\ (mod\ p)$ has a degree and that degree is $<n$. By the induction hypothesis, this congruence has fewer than n solutions. This contradicts the earlier observation that this congruence has at least $n$ solutions. Thus the proof is complete.

\textbf{Corollary 2.27:}
If $b_nx^n+b_{n-1}x^{n-1}+...+b_0\equiv 0\ (mod\ p)$ has more than $n$ solutions, then all the coefficients $b_j$ are divisible by $p$.

\textbf{Theorem 2.28:}\\
If $F(x)$ is a function that maps residue classes $(mod\ p)$ to residue classes $(mod\ p)$, then there is a polynomial $f(x)$ with integral coefficients and degree at most $p-1$ such that $f(x)\equiv F(x)\ (mod\ p)$ for all residue classes $x\ (mod\ p)$.

\textbf{Proof:} Theorem 2.28\\
By Fermat's Congruence we see that
\begin{align}
    1-(x-a)^{p-1}&\equiv 1\ (mod\ p)\text{ if } x\equiv a\ (mod\ p)\\
    1-(x-a)^{p-1}&\equiv 0\ (mod\ p) \text{ otherwise.}
\end{align}
Hence the polynomial
\begin{align}
    f(x)=\sum^p_{i=1}F(i)(1-(x-i)^{p-1})
\end{align}
had the desired properties.

\textbf{Theorem 2.29:}\\
The congruencs $f(x)\equiv 0\ (mod\ p)$ of degree $n$ with leading coefficient $a_n=1$ has $n$ solutions iff $f(x)$ is a factor of $x^p-x$ modulo $p$, that is if and only if $x^p-x=f(x)q(x)+ps(x)$, where $q(x)$ and $s(x)$ have integral coefficients, $q(x)$ has degree $p-n$ and leading coefficient $1$, and where $s(x)$ is a polynomial of degree less than $n$ or $s(x)$ is zero.

\textbf{Proof:} Theorem 2.29\\
First we assume that $f(x)\equiv 0\ (mod\ p)$ has $n$ solutions. Then $n\leq p$ by defintion 2.4. Dividing $x^p-x$ by $f(x)$ we get $x^p-x=f(x)q(x)=r(x)$ where degree $r(x)<n$ or $r(x)<n$ or $r(x)=0$. This equation implies (using Fermat's Theorem) that every solution of $f(x)\equiv 0\ (mod\ p)$ is a solution of $r(x)\equiv 0\ (mod\ p)$. Thus $r(x)\equiv 0\ (mod\ p)$ has at least $n$ solutions and by Corollary 2.27, it follows that every coefficient in $r(x)$ is divisible by $p$, so $r(x)=ps(x)$ as in the Theorem.

Conversely, assume that $x^p-x=f(x)q(x)+ps(x)$ as in the Theorem. By Fermat's Theorem, the congruence $f(x)q(x)\equiv 0\ (mod\ p)$ has $p$ solutions. This congruence has leading term $x^p$. The leading term of $f(x)$ is $x^n$ by hypothesis, and hence the leading term of $q(x)$ is $x^{p-n}$. By Theorem 2.26, the congruence $f(x)\equiv 0\ (mod\ p)$ and $q(x)\equiv 0\ (mod\ p)$ have at most $n$ solutions and $p-n$ solutions, respectively. But every one of the $p$ solutions of $f(x)\equiv 0\ (mod\ p)$ has a solution of at least one of the congruences $f(x)\equiv 0\ (mod\ p)$ and $q(x)\equiv 0\ (mod\ p)$. It follows that the two congruences have exactly $n$ solutions and $p-n$ solutions, respectively.

\textbf{Corollary 2.30:}
If $d|(p-1)$, then $x^d\equiv 1\ (mod\ p)$ has $d$ solutions.

\textbf{Proof:} Corollary 2.30\\
Choose $e$ so that $de=p-1$. Since $(y-1)(1+y+...+y^{e-1})=y^e-1$, on taking $y=x^d$ we see that $x(x^d-1)(1+x^d+...+x^{d(e-1)})=x^p-x$.

Consider
\begin{align*}
    f(x)=(x-1)(x-2)...(x-p+1)
\end{align*}
We assume $p>2$. On expanding, we find that
\begin{align}
    f(x)=x^{p-1}-\sigma_{1}x^{p-2}+\sigma_{2}x^{p-3}-...+\sigma_{p-1}
\end{align}

where $\sigma_j$ is the sum of all products of $J$ distinct members of the set $\{1,2,..,p-1\}$. In the two extreme cases we have $\sigma_1=1+2+3+...+(p-1)=\frac{p-1}{2}$, and $\sigma_{p-1}=1\dot2\dot3\dot...\dot(p-1)=(p-1)!$. The polynomial f(x) has degree $p-1$ and has the $p-1$ roots $1,2,...,p-1\ (mod\ p)$. consequently, the polynomial $xf(x)$ has degree $p$ and has $p$ roots. By Theorem 2.29 in $xf(x)$, we see that there are polynomials $q(x)$ and $s(x)$ such that $x^p-x=xf(x)q(x)+ps(x)$. Since the degree $q(x)=p-p=0$ and leading coefficient $1$, we see that $q(x)=1$. that is, $x^p-x=xf(x)+ps(x)$, which is to say that the coefficients of $x^p-x$ are congruent $mod(\ p)$to those of $xf(x)$. On comparing the coefficients of $x$, we deduce that $\sigma_{p-1}=(p-1)!\equiv -1\ (mod\ p)$, which provides a second proof of Wilson's congruence. On comparing the remaining coefficients, we deduce that $\sigma_{p}\equiv 0\ (mod\ p)$ for $1\leq j\leq p-2$. To these useful observations, we may add one further remark: if $p\geq 5$ then
\begin{align*}
    \sigma_{p-2}\equiv 0\ (mod\ p^2)
\end{align*}
This is Wolstenholme's congruence. To prove it, we note that $f(p)=(p-1)(p-1)...(p-p+1)=(p-1)!$ On taking $x=p$ in (47) we have
\begin{align*}
    (p-1)! = p^{p-1}-\sigma_1p^{p-2}+...+\sigma_{p-3}p^2-\sigma_{p-2}p+\sigma_{p-1}
\end{align*}
We already know that $\sigma_{p-1}=(p-1)!$ On subtracting this amount from both sides and dividing through by $p$, we deduce that
\begin{align*}
    p^{p-2}-\sigma_{1}p^{p-3}+...+\sigma_{p-3}p-\sigma_{p-2}=0
\end{align*}
All terms except the last two contains visible factors of $p^2$. Thus $\sigma_{p-3}p\equiv \sigma_{p-2}\ (mod\ p^2)$. This gives the desired result, since $\sigma_{p-3}\equiv 0\ (mod\ p)$

\textbf{Theorem 3.2:} Gauss' Lemma\\
Let $p$ be an odd prime and $(a,p)=1$.
\begin{align}
    a,2a,3a,...,\frac{p-1}{2}a
\end{align}
and their least positive residues

% Lecture 18

\subsection{Primitive Roots and Power Residues}

\textbf{Definition 2.6:} Order of modulus\\
Let $m>0, m\in\mathbb{Z}$ and $a\in\mathbb{Z}$ such that $(a,m)=1$. Let $h$ be the smallest positive integer such that $a^h\equiv 1\ (mod\ m)$. In this case, we say that the order of $a$ modulo $m$ is $h$, or that $a$ belongs to the exponent $h$ modulo $m$.

Suppose that $a$ has order $h\ (mod\ m)$. Let $k=qh>0$, then $a^k=a^{qh}=(a^h)^q\equiv 1^q\equiv 1\ (mod\ m)$.
Conversely, $k>0$ such that $a^k\equiv 1\ (mod\ m)$, then by the division algorithm, we have $k=qh+r, q\geq0$ and $0\leq r\leq h$. Thus $1\equiv a^k\equiv a^{qh+r}\equiv (a^h)^qa^r\equiv 1^qa^r\equiv a^r\ (mod\ m)$. But $0\leq r<h$ and $h$ is the least positive power of $a$ that is congruent to $1\ (mod\ m)$, so it follows $r=0$. Thus $h|k$ and we have proved that following

\textbf{Lemma 2.31:}\\
If $a$ has order $h\ (mod\ m)$, then the positive integers $k$ such that $a^k\equiv 1\ (mod\ m)$ are precisely those for which $h|k$.

\textbf{Corollary 2.32:}\\
If $(a,m)=1$, then the order of a modulo $m$ divides $\Phi(m)$.

\textbf{Proof:}\\
Each reduced residue class $a$ modulo $m$ has finite order, for by Euler's congruence $a^{\Phi(m)}\equiv 1\ (mod\ m)$. Moreover, if $a$ has order $h$ then by taking $k=\Phi(m)$ in the lemma, we deduce that $h|\phi(m)$.

\textbf{Lemma 2.33:}\\
If $a$ has order $h\ (mod\ m)$, then $a^k$ has order $\frac{h}{(h,k)}\ (mod\ m)$

Note:\\
Since $\frac{h}{(h,k)}=1$ iff $h|k$, we see that Lemma 2.33 contains Lemma 2.31 as a special case.

\textbf{Proof:}\\
Lemma 2.31 says that $(a^k)^J\equiv 1\ (mod\ m)$ iff $h|k_J$. But $h|k_J$ iff $\{\frac{h}{(h,k)}\}|\{\frac{k}{h,k}\}J$. As the divisor is relatively prime to the first factor of the dividend, this relation holds iff $\{\frac{k}{h,k}|J$. Therefore the least positive integer $J$ such that $(a^k)^J\equiv 1\ (mod\ m)$ is $J=\frac{h}{(h,k)}$

Note:\\
If $a$ has order $h\ (mod\ m)$ and b has order $k\ (mod\ m)$ then $(ab)^{hk}=(a^h)^k(b^k)^h\equiv 1\ (mod\ m)$ and using Lemma 2.31 we deduce that the order of $ab$ is a divisor of $hk$.

\textbf{Lemma 2.34:}\\
If $a$ has order $h\ (mod\ m)$ and b has order $k\ (mod\ m)$ and if $(h,k)=1$, then $ab$ has order $hk\ (mod\ m)$.

\textbf{Proof:} Lemma 2.34\\
Let $r=$order of $ab \ (mod\ m)$. We have shown that $r|hk$. We now prove $hk|r$. We note that $b^{rh}\equiv (a^h)^rb^{rh}=(ab)^{rh}\equiv 1 \ (mod\ m)$. Thus $k|rh$ by Lemma 2.31. As $(h,k)=1$, it follows that $k|r$. By a similar argument we see that $h|r$. Using again $(h,k)=1$, we conclude that $hk|r$.

\textbf{Definition 2.7:}Primative Root\\
If $g$ belongs to the exponent $\Phi(m) \ (mod\ m)$, then $g$ is called a \underline{Primitive root}. $(g,m)=1$ and $g^{\Phi(m)}\equiv 1 \ (mod\ m)$ where $\Phi(m)$ is the smallest positive integer with such property.

In view of Lemma 2.31, the number $a$ is a solution of the congruence $x^k \equiv 1 \ (mod\ m)$ iff the order of $a \ (mod\ m)$ divides $k$.

In one special case, Corollary 2.30, we have determined the number of solutions of this congruence. That is, if $p$ is prime and $k|(p-1)$ the there are precicely $k$ residue classes $a \ (mod\ p)$ such that the order of a moulo $p$ is a divisor of $k$. If $k$ happens to be a prime power, we can then determine the exact number of residues $a \ (mod\ p)$ of order $k$.

\textbf{Lemma 2.35:}\\
The divisors of $q^\alpha$ are the numbers $q^\beta$ with $\beta = 0,1,...,\alpha$. Of these, $q^\alpha$ is the only one that is not a divisor of $q^{\alpha-1}$. There are $q^\alpha$ residues $\ (mod\ p)$ of order dividing $q^\alpha$, and among these there are $q^{\alpha-1}$ residues of order dividing $q^{\alpha-1}$. On subtracting we see that there are precisely $q^{\alpha}-q^{\alpha-1}$ residues $a$ of order $q^{\alpha} \ (mod\ p)$


\textbf{Theorem 2.36:}\\
If $p$ is a prime then there exists $\Phi(p-1)$ primitive roots $\ (mod\ p)$

\textbf{Proof:} Theorem 2.36\\
We first establish the existence of at least one primitive root.\\
Let $p-1=p_1^{\alpha_1}p_2^{\alpha_2}...p_J^{\alpha_J}$. By lemma 2.35 we may choose numbers $a_i \ (mod\ p)$ so that $a_i$ has order $p_i^{\alpha_i}$, $i=1,2,...,J$. The numbers $p_i^{\alpha_i}$ are pairwise relatively prime, so by repeated use of lemma 2.34 we see that $g=a_1a_2...a_J$ has order $p_1^{\alpha_1}p_2^{\alpha_2}...p_J^{\alpha_J}=p-1$.\\
That is, $g$ is a primitive root $\ (mod\ p)$.

To complete the proof, we determine the exact number of primitive roots $\ (mod\ p)$.\\
Let $g$ ne a primitive root $\ (mod\ p)$. Then the numbers $g,g^2,...,g^{p-1}$ form a system of reduced residues $\ (mod\ p)$. By Lemma 2.33 we see that $g^k$ has order $(p-1)/(k,p-1)$. Thus $g^k$ is a primitive root iff $(k,p-1)=1$. By definition of Euler's phi function, there are exactly $\Phi(p-1)$ such values of $k$ in the interval $1\leq k\leq p-1$.

\textbf{Definition 2.8:} $n^{th}$ power residue\\
If $(a,p)=1$ and $x^n \equiv a \ (mod\ p)$ has a solution,\\
$a$ is called an \underline{$n^{th}$ power residue modulo $p$}

If $(g,m)=1$ then the sequence $g,g^2,... \ (mod\ m)$ is periodic.\\
If $g$ is a primitive root $\ (mod\ m)$ then the least period of this sequence is $\Phi(n)$ and we see that $g,g^2,...g^{\Phi(m)}$ form a system of reduced residues $\ (mod\ m)$. Thus $g^i \equiv g^J \ (mod\ m)$ iff $i \equiv j \ (mod\ \Phi(m))$. By expressing numbers as powers of $g$, we may convert a multiplicative congruence $\ (mod\ m)$ to an additive congruence $\ (mod\ \Phi(m))$.

\textbf{Theorem 2.37:}\\
If $p$ is a prime and $(a,p)=1$, then the congruence $x^n \equiv a \ (mod\ p)$ had $(n,p-1)$ solutions or no solutions according as
\begin{align*}
    a^{\frac{p-1}{(n,p-1)}} \equiv \ (mod\ p)
\end{align*}
or not.

\textbf{Proof:} Theorem 2.37\\
Let $g$ be a Primative root $\ (mod\ p)$, and choose $i$ such that $g^i \equiv a \ (mod\ p)$. If there is an $x$ such that $x^n \equiv a \ (mod\ p)$ then $(x,p)=1$, so that $x \equiv  g^u \ (mod\ p)$. Thus the proposed congruence is $g^{nu} \equiv g^i \ (mod\ p)$, which is equivalent to $nu \equiv i \ (mod\ p-1)$. Put $k = (m,p-1)$. By Theorem 2.17, this has $k$ solutions if $k|i$, and no solutions if $k\nmid i$.\\
If $k|i$, then $i(p-1)/k \equiv 0 \ (mod\ p-1)$, so that $a^\frac{(p-1)}{k} \equiv g^{i\frac{(p-1)}{k}} = g^{p-1}^{\frac{i}{k}} \equiv 1 \ (mod\ p)$.\\
On the other hand, if $k\nmid i$, then $i(p-1/k \not \equiv (\ (mod\ p-1)))$, and hence $a^{\frac{p-1}{k}}\not \equiv 1 \ (mod\ p)$.

Example: Show that the congruence $x^5 \equiv \ (mod\ 101)$ has 5 solutions.

Solution: By using Binary Algorithm it suffices to verify that $6^20 \equiv 1 \ (mod\ 101)$

\textbf{Corollary 2.38:}Euler's Criterion\\
If $p$ is an off prime and $(a,p)=1$, then $x^2 \equiv a \ (mod\ p)$ has two solutions or no solutions according as $a^\frac{p-1}{z} \equiv 1 \ (mod\ p)$ or $a^\frac{p-1}{2} \equiv -1 \ (mod\ p)$.

\textbf{Proof:} Corollary 2.38 \\
Put $b=a^{\frac{(p-1)}{2}}$. Thus $b^2=a^{p-1} \equiv 1 \ (mod\ p)$ by Fermat's congruence. From lemma 2.10, it follows that $b \equiv \pm 1 \ (mod\ p)$.\\
If $b \equiv -1 \ (mod\ p)$, then $x^2 \equiv a \ (mod\ p)$ has no solutions, by Theorem 2.37.
If $b \equiv 1 \ (mod\ p)$, then $x^2 \equiv a \ (mod\ p)$ has 2 solutions, by Theorem 2.37.

Primitive roots is a tool for analyzing certain congruences $\ (mod\ p)$. What can be said about other moduli?

\textbf{Theorem 2.39:}\\
If $p$ is a prime then there exists $\Phi(\Phi(p^2))=(p-1)\Phi(p-1)$ primitive roots $\ (mod\ p^2)$.

\textbf{Proof:} Theorem 2.39\\
Exercise

\textbf{Theorem 2.40:}\\
If $p$ is an odd prime and $g$ is a primitive root $\ (mod\ p^2)$, then $g$ is a primitive root $\ (mod\ p^{\alpha})$ for $\alpha = 3,4,5...$

\textbf{Proof:} Theorem 2.40\\
The prime $p=2$ must be excluded, for $g=3$ is a primitive root $\ (mod\ 4)$, but not $\ (mod\ 8)$. Indeed it is easy to verify that $a^2 \equiv 1 \ (mod\ 8)$, for any odd number $a$. As $\Phi(8)=4$, it follows that there is no primitive root $\ (mod\ 8)$.\\
Suppose that $a$ is odd. Since $8|(a^2-1)$ and $2|(a^2+1)$, it follows that $16|(a^2-1)(a^2+1)=a^4-1$. That is $a^4 \equiv 1 \ (mod\ 16)$. On repeating this argument we see that $a^8 \equiv 1 \ (mod\ 32)$, and in general that $a^2^{\alpha-2} \equiv 1 \ (mod\ a^{\alpha})$ for $\alpha \geq 3$. Since $\Phi(2^{\alpha})=2^{\alpha-1}$ we conclude that if $\alpha \geq 3$ then
\begin{align*}
     a^\frac{\Phi(2^{\alpha})}{2} \equiv 1 \ (mod\ 2^{\alpha})
\end{align*}
for all odd $a$, and hence that there is no primitive root $\ (mod\ 2^{\alpha})$ for $\alpha=3,4,5...$

Suppose that $p$ is an odd prime and that $g$ is a primitive root $\ (mod\ p^{\alpha})$. We may suppose that $g$ is odd, for if $g$ is even then we have only to replace $g$ by $g+p^{\alpha}$, which is odd. The numbers $g,g^2,...,g^{\Phi(p^{\alpha})}$ forms a reduced residue system $\ (mod\ p^{\alpha})$. Since these numbers are odd, they also form a reduced residue system $\ (mod\ 2p^{\alpha})$. Thus $g$ is a primitve root $\ (mod\ 2p^{\alpha})$.

We have established that a primitive root exists $\ (mod\ m)$ when $m=1,2,4,p^{\alpha}$ or $2p^{\alpha}$, ($p$ odd prime) but that there is no primitive root $\ (mod\ 2^{\alpha})$ for $\alpha\geq 3$. Suppose now that $m$ is not a prime power or twice a prime power. Then $m$ can be expressed as a product, $m=m_1m_2$ with $(m_1,m_2)=1$, $m_1>2$, $m_2>2$. Let $e=lcm(\Phi(m_1),\Phi(m_2))$. If $(a,m)=1$ then $(a,m_1)=1$ so that $a^{\Phi(m_1)} \equiv 1 \ (mod\ m_1)$, and hence $a^e \equiv 1 \ (mod\ m_1)$. Similarly $a^e \equiv 1 \ (mod\ m_2)$, and hence $a^e \equiv 1 \ (mod\ m)$.
Since $2|\Phi(n)$ for all $n>2$, we see that $2|(\Phi(m_1),\Phi(m_2))$, so that by Theorem 1.13:
\begin{align*}
    e = \frac{\Phi(m_1)\Phi(m_2)}{(\Phi(m_1),\Phi(m_2))} < \Phi(m_1)\Phi(m_2) = \Phi(m)
\end{align*}
Thus there is no primitive root in this case.

So we then have...

\textbf{Theorem 2.41:}\\
There exists a primitive root $\ (mod\ m)$ iff $m=1,2,4,p^{\alpha}$ or $2p^{\alpha}$, where $p$ is an odd prime.\\
Theorem 2.37 and its proof generalises to any modulus $m$ possessing a primitive root.

\textbf{Corollary 2.42:}\\
Suppose that $m=1,2,4,p^{\alpha}$ or $2p^{\alpha}$, where $p$ is odd prime. If $(a,m)=1$ then the congruence $x^n \equiv a \ (mod\ m)$ has $(n,\Phi(m))$ solutions or no solutions, according as
\begin{align*}
    a^{\frac{\Phi(m)}{(n,\Phi(m))}} \equiv 1 \ (mod\ m)
\end{align*}
or not.

Example:\\
Determine the number of solutions of the congruence $x^4 \equiv 61 \ (mod\ 117)$

Solution:\\
We note that $117 = 3^2\dot 13$
As $\Phi(9)/(4,\Phi(9))=6/(4,6)=3$ and $61^3 \equiv (-2)^3 \equiv 1 \ (mod\ 9)$ we deduce that the congruence $x^4 \equiv 61 \ (mod\ 9)$ has $(4,\Phi(9))=2$ solutions.\\
Similarly $\Phi(13)/(4,\Phi(13))==3$ and $61^3 \equiv (-4)^3 \equiv  1 \ (mod\ 13)$ so the congruence $x^4 \equiv 61 \ (mod\ 13)$ has $(4,\Phi(13))=4$ solutions.\\
Thus by Theorem 2.20, the number of solutions $\ (mod\ 117)$ is $2\dot4 = 8$.

This method fails in case the modulus is divisible by $8$, as corollary $2.42$ does not apply to higher powers of $2$.

For this we have...

\textbf{Theorem 2.43:}\\
Suppose that $\alpha\geq 2$. The order of $5 \ (mod\ 2^\alpha)$ is $2^{\alpha-2}$. The numbers $\pm 5,\pm 5^2,...,\pm 5^2^{\alpha-2}$ form a system of reduced residues $\ (mod\ 2^{\alpha})$. If $a$ is odd, then there exists $i$ and $J$ such that $a \equiv  (-1)^i5^J \ (mod\ 2^{\alpha})$. The values of $i$ and $J$ are uniquely determined $\ (mod\ 2)$ and $(\ (mod\ 2^{\alpha-2}))$, respectively.

\textbf{Corollary 2.44:}\\
Suppose $\alpha \geq 3$ and that $a$ is odd.\\
If $n$ is odd, then the congruence $x^n \equiv  a \ (mod\ 2^{\alpha})$ has exactly one solution.\\
If $n$ is even, then choose $\beta$ such that $(n,2^{\alpha-2})=2^\beta$. The congruence $x^n \equiv a \ (mod\ 2^{\alpha})$ has $2^{\beta+1}$ solutions or no solutions according as $a \equiv  1 \ (mod\ 2^{\beta+2})$ or not.

% lecture 19

\section{Quadratic Reciprocity and Quadratic Forms}

% Lecture 19

\subsection{Quadratic Residues}

\subsection{Quadratic Reciprocity}

% Lecture 20

\textbf{Theorem 3.4:}\\

\begin{align}
    \frac{p}{q}\frac{q}{p}=(-1)^{\frac{p-1}{2}\frac{q-1}{2}}
\end{align}

Note: If $p$ and $q$ are distinct odd primes of the form $4k+3$, then one of the congruences $x^2\equiv p\ (mod\ q)$ or $x^2\equiv q\ (mod\ p)$ is a solutions and the other is not. However, if at least one of the primes is of the for $4k+3$, then both congruences are soluable or both are not.

\textbf{Proof:} Theorem 3.4\\
Let $S$ be the set of pairs of of integers $(x,y)$ such that $1\leq x\leq\frac{p-1}{2}$ and $1\leq y\leq\frac{q-1}{2}$.

The set $S$ has $\frac{(p-1)(q-1)}{4}$ elements. Seperate this set into two mutually exclusive subsets $S_1$ and $S_2$ according $qx>py$ or $qx<py$. Note that there are no pairs $(x,y)\in S$ such that $qx=py$.

The set $S_1$ can be described as the set of all pairs $(x,y)$ such that
\begin{align}
    1\leq x\leq\frac{p-1}{2},\quad 1\leq y\leq\frac{qx}{p}
\end{align}
The number of pairs in $S_1$ is
\begin{align}
    \sum^\frac{p-1}{2}_{x=1}[\frac{qx}{p}]
\end{align}
Similarly for $S_2$ the number of pairs in $S_2$ is
\begin{align}
    \sum^\frac{p-1}{2}_{y=1}[\frac{qy}{p}]
\end{align}

Thus we have:
\begin{align}
     &\sum^{\frac{p-1}{2}}_{j=1}[\frac{qj}{p}]+\sum^{\frac{q-1}{2}}_{j=1}[\frac{pj}{q}]\\
    =&\frac{p-1}{2}\frac{q-1}{2}
\end{align}
and hence
\begin{align}
    \frac{p}{q}\frac{q}{p}=(-1)^{\frac{p-1}{2}\frac{q-1}{2}}
\end{align}

\textbf{Example:} Compute $(\frac{42}{61})$

...

\subsection{The Jacobi Symbol}

% Lecture 21

\subsection{Binary Quadratic Forms}

\subsection{Sums of Two Squares}







\end{document}
