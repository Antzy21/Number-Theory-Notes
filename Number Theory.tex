\documentclass[a4paper]{article}

\usepackage{amsmath}
\usepackage{amsfonts}
\usepackage{amssymb}
\usepackage{enumitem}
\setlength{\parindent}{0em}
\setlength{\parskip}{1em}
\renewcommand{\baselinestretch}{1}
\title{Number Theory: Lecture Notes}
\author{Anthony Dunford \and Chris Nash}
\begin{document}

\maketitle

% Lecture 1

\section{Divisibility and Primes}

\subsection{Introduction}

\textbf{Well ordering Principle:}\\
Let $S\neq0$ be a set of positive integers.\\
Then there exists $s\in S$ such that for all $a\in S, s\leq a$

\textbf{Induction:}\\
If a set $s$ of positive integers contains the integer $1$\\
And contains $n+1$ whenever it contains $n$\\
Then $S$ consists of all the positive integers

\subsection{Divisibility}

\textbf{Definition 1.1:} Divisibility\\
An integer $b$ is divisible by and integer $a\neq0$ if there is an integer
$x$ such that $b=ax$.\\s
We write $a|b$ (a divides b)

\textbf{Theorem 1.1:} Properties of divisibility
\begin{enumerate}
    \item $a|b\            \rightarrow a|bc\quad c\in \mathbb{Z}$
    \item $a|b\ \&\ b|c    \rightarrow a|c$
    \item $a|b\ \&\ a|c    \rightarrow a|(bx+cy)\quad x,y\in\mathbb{Z}$
    \item $a|b\ \&\ b|a    \rightarrow a=\pm b$

% Lecture 2

    \item $a|b,\ a>0,\ b>0 \rightarrow a\leq b$
    \item $m\neq0,\ a|b\   \leftrightarrow ma|mb$
\end{enumerate}

\textbf{Proof:} Theorem 1.1 (3)

$a|b \rightarrow b=ar$ for some $r\in \mathbb{Z}$
and $a|c \rightarrow c=as$ for some $s\in \mathbb{Z}$
Hence $bx+cy=a(rx+sy)$ and this proves that $a|(bx+cy)$


\textbf{Theorem 1.2:} The Division Algorithm\\
Let $a,b\in\mathbb{Z},\ a>0$.\\
Then there exists unique $q,r\in\mathbb{Z}$ such that $b=qa+r,\ 0\leq r<a$.\\
If $a\nmid b$ then $0<r<a$

\textbf{Proof:} Theorem 1.2

Consider the arithmetic progression:

$...,b-3a,b-2a,b-a,b,b+a,b+2a,b+3a,...$

In the sequence select the smallest non-negative member and denote it by
$r$. Thus by definition $r$ satisfies the inequalities of the theorem. But
also $r$, being in the sequence, is of the form $b-qa$, and thus q is defined
in terms of $r$.

To prove uniqueness we suppose there is another pair $q_1$ and $r_1$ satisfying
the same conditions. First we prove that $r=r_1$. If not, we may presume
that $r<r_1$ so that $=<r_1-r<a$ and then we see that $r_1-r=a(q-q_1)$ and
so $a|(r_1-r)$, a contradiction to Theorem 1.1 (5). Hence $r=r_1$ and also
$q=q_1$.

Note: We stated the theorem with $a>0$. However this is not necessary and
we may formulate as:

Given $a,b\in\mathbb{Z}$ , $a\neq 0$ , there exists $q,r\in\mathbb{Z}$ such
that $b=qa+r$ , $0\leq r < |a|$.


\textbf{Definition 1.2:}\\
The integer $a$ is a \underline{common divisor} of $b$ and $c$ if $a|b$,
$a|c$
and at least $b\neq0$ or $c\neq0$, the greatest among their common divisors
is
called the \underline{greatest common divisor} of $b$ and $c$ and is denoted
by
$gcd(b,c)$ or $(b,c)$.

Let $b_1,...,b_n\in\mathbb{Z}$, not all zero.
We denote $g=(b_1,...b_n)$ to be the greatest common divisor.

\textbf{Theorem 1.3:}\\
If $g=(b,c)$, then there exist $x_0,y_0\in\mathbb{Z}$ such that $g=(b,c)=bx_0+cy_0$

\textbf{Proof:} Theorem 1.3

Consider the linear combination $bx+cy$, where $x,y$ range over all the integers.
This set of integers \{$bx+cy$\} includes positive and negative values and
also 0. ($x=y=0$). Choose $x_0$ and $y_0$ so that $bx_0+cy_0$ is the least
positive integer $l$ in the set. Thus $l=bx_0+cy_0$.

Next we prove that $l|b$ and $l|c$. Assume that $l \nmid b$ , then it follows
that there exists integers $q$ and $r$ , by Theorem 1.2, such that $b=lq+r$
with $0<r<l$. Hence we have $r=b-lq=b-q(bx_0+cy_0)=b(l-qx_0)+c(-qy_0)$, and
thus $r$ is in the set \{$bx+cy$\}. This contradicts the fact that $l$ is
the least positive integer in \{$bx+cy\}$. Similar proof for $l|c$. Now since
$g=(b,c)$ we may write $b=gB$ , $c=gC$ and $l=bx_0+cy_0=g(Bx_0+Cy_0)$. Thus
$g|l $ and so by Theorem 1.1 (5) we conclude that $g \leq l$. We know $g<l$
is impossible since $g$ is the greatest common divisor, so $g=l=bx_0+cy_0$.

% Lecture 3

\textbf{Theorem 1.4:}\\
The greatest common denominator of $b$ and $c$ can be characterised in the
following two ways:
\begin{enumerate}
    \item It is the least positive value of $bx+cy$ where $x,\ y\in\mathbb{Z}$
    \item If $d$ is any common divisor of $b$ and $c$ then $d|g$ by Theorem
1.1 (3).
\end{enumerate}

\textbf{Proof:} Theorem 1.4

\begin{enumerate}
\item Follows from Theorem 1.3
\item If $d$ is any common divisor of $b$ and $c$, then $d|g$ by Theorem
1.1 (3). Moreover, there cannot be two distinct integers with property (2),
because of Theorem 1.1 (4).
\end{enumerate}

Note: If $d=bx+cy$ , then $d$ is not necessary the $gcd(b,c)$. However, it
does follow from such align that $(b,c)$ is a divisor of $d$. In particular
, if $bx+cy=1$ for some $x,y\in\mathbb{Z}$ , then $(b,c)=1$.

\textbf{Theorem 1.5:}\\
Given $b_1,...,b_n\in\mathbb{Z}$ not all zero with greatest common divisor
$g$,
there exists integers $x_1,...,x_n$,  such that
\begin{align}
    g=(b_1,...,b_n)=\sum^n_{j=1}b_jx_j
\end{align}
Furthermore, g is the least positive value of the linear form $\sum^n_{j=i}b_jy_j$
where the $y_j$ runs over all integers; also $g$ is the positive common
divisor of $b_1,...,b_n$ that is divisible by every common divisor.

\textbf{Proof:} Theorem 1.5

Exercise for the reader.



\textbf{Theorem 1.6:}\\
For any $m\in\mathbb{Z}, m>0$
\begin{align}
    (ma,mb)=m(a,b)
\end{align}

\textbf{Proof:} Theorem 1.6

By Theorem 1.4 we have:

$(ma,mb)$ = least positive value of $max+mby$
=$m$ \{ least positive integer of $ax+by$\}
= $m(a,b)$


\textbf{Theorem 1.7:}\\
If $d|a$, $d|b$ and $d>0$, then
\begin{align}
    \bigg(\frac{a}{d},\frac{b}{d}\bigg)=\frac{1}{d}(a,b)
\end{align}
If $(a,b)=g$, then
\begin{align}
    \bigg(\frac{a}{g},\frac{b}{g}\bigg)=1
\end{align}



\textbf{Proof:} Theorem 1.7

The second assertion is the special case of the first using $d=(a,b)=g$.
The first assertion is a direct consequence of Theorem 1.6, obtained by
replacing $m,a,b$ in Theorem 1.6 by $d,\frac{a}{d},\frac{b}{d}$ respectively.

\textbf{Theorem 1.8:}\\
If $(a,m)=(b,m)=1$ then $(ab,m)=1$

\textbf{Proof:} Theorem 1.8

Exercise for the reader.

\textbf{Definition:} 1.3

We say that $a$ and $b$ are \underline{relatively prime} in case $(a,b)=1$,
and that $a_1,a_2,...,a_n$ are relatively prime in the case $(a_1,a_2,...,a_n)=1$.
We say that $a_1,a_2,...,a_n$ are \underline{relatively prime in pairs} in
case $(a_i,a_j)=1$ for all $i=1,2,...,n$ and $j=1,2,...n$ with $i \neq j$.

Note: $(a,b)=1$ we also say $a$ and $b$ are \underline{coprime}.

\textbf{Theorem 1.9:}\\
For any $x\in\mathbb{Z}$ we have
\begin{align}
(a,b)=(b,a)=(a,-b)=(a,b+ax)
\end{align}

\textbf{Proof:} Theorem 1.9

Exercise for the reader.

\textbf{Theorem 1.10:} Euclid's Lemma\\
If $c|ab$ and $(b,c)=1$, then $c|a$.

\textbf{Proof:} Theorem 1.10

By Theorem 1.6 , $(ab,ac)=a(b,c)=a$. By hypothesis $c|ab$ and clearly $c|ac$,
so $c|a$ by Theorem 1.4 (2).

Now we observe for $c\neq 0$ , we have $(b,c)=(b,-c)$ by Theorem 1.9 and
hence we may presume $c>0$.

\textbf{Theorem 1.11:} The Euclidean Algorithm\\
Given $b,c\in\mathbb{Z}, c>0$, we can make a repeated
application of the division algorithm, \textbf{Theorem 1.2},
to obtain a series of aligns
\begin{align}
b=cq_1+r_1          & \quad0<r_1<c\\
c=r_1q_2+r_2        & \quad0<r_2<r_1\\
r_1=r_2q_3+r_3      & \quad0<r_3<r_2\\
...\\
r_j=r_{j+1}q_j+r_j  & \quad0<r_j<r_{j-1}\\
r_{j-1}=r_jq_{j+1}.
\end{align}
The greatest common divisor $(b,c)$ of $b$ and $c$ is $r_j$,
the last nonzero remainder in the division process.
Values of $x_0$ and $y_0$ in $(b,c)=bx_0+cy_0$
can be obtained by writing each $r_i$ as a linear combination
of $b$ and $c$.

\textbf{Proof:} Theorem 1.11

See Theorem 1.11 in the textbook or Theorem 1.13 in the Lecture Notes.

% Lecture 4

\textbf{Example 1}
$gcd(841,160)$
\begin{align}
\begin{split}
841&=160\times5 + 41 \\
160&=41\times3 + 37 \\
41&=37\times 1 + 4 \\
37&=34\times 9 + 1 \\
4&=1\times 4 + 0 \\
\end{split}
\end{align}

Hence (841,160)=1 working backwards gives:


\begin{align}
1&=37\times1 - 4\times9 \\
1&=37\times1 - (41-37)\times9 \\
1&=37\times10 - 41\times9 \\
1&=(160-3\times41)\times10 - 41 \times 9 \\
1&=160\times10 - 41\times39 \\
1&=160\times10 - (841-160\times5)\times39 \\
1&=(-39)\times841 + 205\times160 \\
\end{align}

Note the solution is not unique:
\begin{align}
1=121\times841 - 636\times 160
\end{align}

\textbf{Example 2} Extended Algorithm


\begin{align}
\begin{split}
r_i&=r_{i-2} - q_ir_{i-1} \\
x_i&=x_{i-2} - q_ix_{i-1} \\
y_i&=y_{i-2} - q_iy_{i-1} \\
r_1&=b , r_0=c \\
x_1&=1 , x_0=0 \\
y_1&=0 , y_0=1 \\
\end{split}
\end{align}



We want to compute the $gcd(841,160)$ and express as a linear combination
of 841 and 160.


\textbf{Definition 1.4:}\\
The integers $a_1,...,a_n$, all different from zero, have a
\textbf{common multiple} $b$ if $a_i|b$ for $i=1,...,n$. The least of
the positive common multiples is called the
\textbf{least common multiple} and it is denoted by
$[a_1,...,a_n]$ or $lcm(a1,...,a_n)$

\textbf{Theorem 1.12:}\\
If $b$ is any common multiple of $a_1,...,a_n$,
then $[a_1,...,a_n]\ |\ b$. This is the same as saying that if $h=[a_1,...,a_n]$
then $0,\pm h,2\pm h,...$ comprise all the common multiples of $a_1,...,a_n$.

\textbf{Proof:} Theorem 1.12

Let $m$ be any common multiple and divide $m$ and $h$. By Theorem 1.2 , $\exists
q,r$ such that $m=qh+r$ , $ 0\leq r<h.$ We must probe that $r=0$. If $r\neq0$
we argue as follows. For each $i=1,2,...,n$ we know that $a_i|h$ and $a-i|m$
, so that $a_i|r$ . Thus $r$ is a positive common multiple of $a_1,a_2,...,a_n$
contrary to the fact that h is the least of all positive common multiples.


\textbf{Theorem 1.13:}\\
If $m>0$
\begin{enumerate}
    \item $[ma,mb] = m[a,b]$
    \item $[a,b](a,b)=|ab|$
\end{enumerate}

\textbf{Proof:} Theorem 1.13

\begin{enumerate}
\item Let $H=[ma,mb]$ and $h=[a,b]$. Then $mh$ is a multiple of $ma$ and
$mb$, so that $mh\geq H$. Also, $H$ is a multiple of both $ma$ and $mb$ so
$H/m$ is a multiple of $a$ and $b$. Thus, $H/m \geq h$ from which it allows
that $mh=H$.

\item It will suffice to prove this for $a,b\in\mathbb{Z}$ with $a>0,b>0$
, since $[a,-b]=[a,b]$. We begin with the special case where $(a,b)=1$. Now
$[a,b]=1$, is a multiple of a, say $ma$. Then $b|ma$ and $(a,b)=1$, so by
Theorem 1.10 we conclude that $b|m$. Hence $b \leq m$ , $ba\leq ma$. But
$ba$, being a positive common multiple of 4b4 and $a$ , cannot be less tahn
the least common multiple, so $ba=ma=[a,b]$.

Let $(a,b)=g>1$. we have $(a/g,b/g)=1$ by Theorem 1.7. Applying the result
of the previous paragraph we have:
\begin{align}
[a/g,b/g]\cdot(a/g,b/g)=ab/g
\end{align}

Multiplying by $g^2$ and using Theorem 1.6 as well as the first part (1.),
we get $[a,b]\cdot(a,b)=ab$.
\end{enumerate}

% Up to where Chris finished proofs and Examples

\subsection{Primes}

\textbf{Definition 1.5:}\\
An integer $p>1$ is called a \textbf{prime number} if there is no divisor
$d$ of $p$ satisfying $1<d<p$. If an integer $a>1$ is not a prime, is is
called a \textbf{composite number}.

\textbf{Theorem 1.14:}\\
Every integer $n>1$ can be expressed as a product of primes
(with perhaps only one factor).

% Lecture 5

\textbf{Theorem 1.15:}\\
If $p|ab$, p prime, then $p|a$ or $p|b$.
More generally if $p|a_1...a_n$, then $p$ divides at least on of the factors $a_i$
If $p\nmid a$, then $(a,p)=1$ and so by \textbf{Thm 1,10}, $p|b$.
For the general case, we use induction.

\textbf{Theorem 1.16:} Fundamental Theorem of Arithmatic\\
The factoring of any integer $n>1$ into primes is unique apart from
the order of the prime factors.

\textbf{Definition 1.6:}\\
We call $a$ a square (or \textbf{perfect square}) if it can be written as $a=n^2$.
By the \textbf{F.T.A.} $a$ is a square if all the exponents $\alpha(p)$ in (1.6)
are even. We say that $a$ is \textbf{square free} if 1 is the largest square
dividing $a$. Thus $a$ is square free iff the exponents $\alpha(p)=0$ or $1$
If p is prime, then the assertion $p^k||a$ is equivalent to $k=\alpha(p)$.

\textbf{Theorem 1.17:} (Euclid)\\
The number of primes is infinite.

\textbf{Definition 1.7:}\\
Let $n\in\mathbb{N}$ and $p$ a prime. Then
\begin{align}
    v_p(n) = max(k\in\mathbb{N}_{\&()}:p^k|n)
\end{align}
where k is the unique non-negative integer such that $p^k|n$ but $p^{k+1}|n$\\
Equivalently $V_p(n)=k$ iff $n=p^kn'$ where $n'\in\mathbb{N}$ and $p\nmid n'$

\textbf{Lemma:}
Let $n,m\in\mathbb{N}$ and $p$ be a prime. then
\begin{align}
    v_p(mn)=v_p(m)+v_p(n)
\end{align}

% Lecture 6

\section{Congruences}

\subsection{Congruences}

\textbf{Definition 2.1:}\\
If $m\in\mathbb{Z}$, $m\neq0$ is such that $m|a-b$, we say that $a$ is
\underline{congruent to} $b$ modulo $m$ and we write $a\equiv b\ (mod\ m)$

Since $a-b$ is divisible by $-m$, we can socus our attention to a positive modulus.
We will assume in this chapter that $m>0$.

\textbf{Theorem 2.1:} Properties of Congruences
\begin{enumerate}
    \item $a\equiv b\ (mod\ m)$ $b\equiv a\ (mod\ m)$,
    and $a-b\equiv 0\ (mod\ m)$ are equivalent statements.
    \item If $a\equiv b\ (mod\ m)$ and $b\equiv c\ (mod\ m)$,
    then $a\equiv c\ (mod\ m)$
    \item If $a\equiv b\ (mod\ m)$ and $c\equiv d\ (mod\ m)$,
    then $a+c\equiv b+d\ (mod\ m)$
    \item If $a\equiv b\ (mod\ m)$ and $c\equiv d\ (mod\ m)$,
    then $ac\equiv bd\ (mod\ m)$
    \item If $a\equiv b\ (mod\ m)$ and $d|m,\ d>0$,
    then $a\equiv b\ (mod\ d)$
    \item If $a\equiv b\ (mod\ m)$
    then $ac\equiv bc\ (mod\ mc)$ for $c>0$
\end{enumerate}

\textbf{Theorem 2.2:}\\
Let $f$ denote a polynomial with integral coefficients. If $a\equiv b\ (mod\ m)$
then $f(a)\equiv f(b)\ (mod\ m)$

\textbf{Theorem 2.3:}
\begin{enumerate}
    \item If $ax\equiv by\ (mod\ m)$ and $x\equiv y\ (mod\ frac{m}{(a,m)})$
    \item $ax\equiv by\ (mod\ m)$ and $(a,m)=1$, then $x\equiv y\ (mod\ m)$
    \item $x\equiv y\ (mod\ m_i)$ for $i=1,...,r$ iff $x\equiv y\ (mod\ [m_1,...,m_r)$
\end{enumerate}

% Lecture 7

\textbf{Definition 2.2:}\\
If $x\equiv y\ (mod\ m)$ then y is called a \underline{residue} of $x\ (mod\ m)$.
A set $x_1,...,x_m$ is called a \underline{complete residue system modulo $m$} if
for every integer $y$, there is one and only one $x_j$ such that $y=x_j\ (mod\ m)$

\textbf{Theorem 2.4:}\\
If $b\equiv c\ (mod\ m)$, then $(b,m)=(c,m)$.

\textbf{Definition 2.3:}\\
A \underline{reduced residue system} modulo $m$ is a set of integers $r_i$ such
that\\ $(r_i,m)=1,\ r_i\not\equiv r_j,\ (mod\ m)$ if $i\neq j$, and such that every
$x$ prime to $m$ (coprime) is congruent modulo $m$ to some member $r_i$ of the set.
\begin{itemize}
    \item You can obtains a reduced residue system by deleting from a complete
    residue system modulo
    $m$ those members that are not relatively prime to $m$.
    \item We will denote by $\Phi(m)$ to be the number of elements of a reduced
    residue system modulo $m$.
    \item All reduced reside system modulo $m$ have the same number of elements.
    \item $\Phi(m)$ is called the \underline{Euler's $\Phi$-function} or
    \underline{Euler's totient-function}
\end{itemize}

\textbf{Theorem 2.5:}\\
The number $\Phi(m)$ is the number of positive integers less than or equal to
$m$ are relatively prime to $m$.

\textbf{Theorem 2.6:}\\
Let $(a,m)=1$. Let $r_1,...,r_n$ be a complete, or a reduced,
residue system modulo $m$. Then $ar_1,...,ar_n$ is a complete,
or a reduced, residue system, respectively, modulo $m$.

% Lecture 8

\textbf{Theorem 2.7:} Fermat's Theorem\\
Let $p$ denote a prime. If $p\nmid a$ then\\
$a^{p-1}\equiv 1\ (mod\ p)$. For every integer $a$,\\
$a^{p}\quad\equiv a\ (mod\ p)$.

\textbf{Theorem 2.8:} Euler's Generalization of Fermat's Theorem\\
If $(a,m)=1$, then
\begin{align}
    a^\phi(m)\equiv 1\ (mod\ m)
\end{align}

\textbf{Theorem 2.9:}\\
If $(a,m)=1$ then there is an $x$ such that $ax\equiv 1\ (mod\ m)$.
Any two such $x$ are congruent $(mod\ m)$.
If $(a,m)>1$ then there is no such $x$.

\textbf{Lemma 2.10:}\\
Let p be a prime number.
Then $x^2\equiv 1\ (mod\ p)$ iff $x\equiv\pm 1\ (mod\ p)$.

\textbf{Theorem 2.11:} Wilson's Theorem\\
If $p$ is prime, then $(p-1)!\equiv -1\ (mod\ p)$

\textbf{Theorem 2.12:}\\
Let $p$ denote a prime. Then $x^2\equiv -1\ (mod\ p)$ has solutions iff
$p=2$ or $p\equiv 1\ (mod\ 4)$.

% Lecture 9

\textbf{Proof:} Theorem \\

\textbf{Theorem 2.13:}\\
If $p$ is prime and $p\equiv 1\ (mod\ 4)$, then there exists positive integers
$a$ and $b$ such that $a^2+b^2=p$.

% Lecture 10

\textbf{Lemma 2.14:}\\
Let $q$ be a prime factor of $a^2+b^2$.
If $q\equiv 3\ (mod\ 4)$ then $q|a$ and $q|b$.


\textbf{Theorem 2.15:} (Fermat)\\
Let
\begin{align}
    n =  2^\alpha\prod_{p\equiv 1(4)}p^\beta\prod_{q\equiv 3(4)}q^\gamma
\end{align}
Then $n$ can be expressed as a sum of two squares iff all the exponents of
$\gamma$ are even.

\subsection{Solutions of Congruences}
\begin{itemize}
    \item Let $f(x)$ denote a polynomial, e.g.
    \begin{align}
        f(x)=a_nx^n+...+a_0
    \end{align}
    \item if $u\in\mathbb{Z}$ such that $f(u)\equiv0\ (mod\ m)$ then we say
    that $u$ is a\\ \underline{solution of the congruence} $f(x)\equiv0\ (mod\ m)$
    \item If $u$ is a solution of $f(x)\equiv0\ (mod\ m)$ and if
    $v\equiv u\ (mod\ m)$, then theorem 2.2 shows that $v$ is also a solution.
    \begin{itemize}
        \item $x\equiv u\ (mod\ m)$ is a solution of $f(x)\equiv0\ (mod\ m)$
        meaning that every integer congruent to $u$ modulo $m$ satisfied
        $f(x)\equiv0\ (mod\ m)$.
    \end{itemize}
\end{itemize}

\textbf{Definition 2.4:}\\
Let $r_1,...,r_m$ denote a complete residue system modulo $m$.\\
The \underline{number of solutions} of $f(x)\equiv0\ (mod\ m)$ is the number of
the $r_i$ such that $f(r_i)\equiv0\ (mod\ m)$

% Lecture 11

\textbf{Definition 2.5:}\\
Let $f(x)=a_nx^n+...+a_0$. If $a_n\equiv0\ (mod\ m)$ the
\underline{degree of the congruence} $f(x)\equiv0\ (mod\ m)$ is $n$.
If $a_n\equiv0\ (mod\ m)$, let $j$ be the largest integer such that
$a_j\not\equiv0\ (mod\ m)$; then the degree of the congruence is $j$.
If there is no such integer $j$, then no degree is assigned to the congruence.

\textbf{Theorem 2.16:}\\
If $d|m$, $d>0$, and if $u$ is a solution of $f(x)\equiv0\ (mod\ m)$, then $u$
is a solution of $f(x)\equiv0\ (mod\ d)$
\begin{itemize}
    \item We say that $f(x)\equiv0\ (mod\ m)$ is an
    \underline{identical congruence} if it holds for all integers $x$
    \begin{itemize}
        \item If $f(x)$ is a polynomial whose coefficients are divisible by $m$,
        then $f(x)\equiv0\ (mod\ m)$ is an identical congruence
        \item e.g. $x^p\equiv x\ (mod\ p)$ is true for all integers $x$ by theorem 2.5
    \end{itemize}
\end{itemize}

\textbf{Theorem 2.17:} Linear Congruences\\
Let $a,b$ and $m>0$ be given integers, and put $g=(a,m)$.
The congruence $ax\equiv b\ (mod\ m)$ has a solution iff $g|b$.
If this condition is met, then the solution forms an arithmetic progression
with common difference $\frac{m}{g}$, giving $g$ solutions $(mod\ m)$.

\textbf{How to solve general linear congruences:}
Let $a,b\in\mathbb{Z}$ and let $n\in\mathbb{N}$.
Suppose we wish to solve the linear congruence
\begin{align}
    ax\equiv b\ (mod\ n)
\end{align}
Firstly apply the Extended Euclidean Algorithm to compute $d=gcd(a,n)$ and find
$x',y'\in\mathbb{Z}$ such that
\begin{align}
    ax'+ny'=d
\end{align}
If $d\nmid b$ then there are no solutions by theorem 2.17. Otherwise, there
are exactly $d$ solutions modulo $n$ by theorem 2.17, which we can find as follows.

Write
\begin{align}
    a=da',\quad b=db',\quad n=dn'
\end{align}
Dividing (18) by $d$ gives
\begin{align}
    a'x'+n'y'=1
\end{align}
Thus reducing mod $n'$ gives $a'x'\equiv1\ (mod\ n')$ and multiplying by $b'$
gives $a'(b'x')\equiv b'\ (mod\ n')$. Therefore $t:=b'x'$ is the unique solution
to $a'x\equiv b'\ (mod\ n')$. Now by theorem 2.17 the solutions to (17) are
$t,t+n',...,t+(d-1)n'$

% Lecture 12

\subsection{The Chinese Remainder Theorem}
Solve Simultaneous Congruences

Find x (is there are any) that satisfies
\begin{align}
    \begin{split}
        x & \equiv a_1\ (mod\ m_1)\\
        x & \equiv a_2\ (mod\ m_2)\\
          & ...\\
        x & \equiv a_r\ (mod\ m_r)\\
    \end{split}
\end{align}

\textbf{Theorem 2.18:} The Chinese Remainder Theorem\\
Let $m_1,...,m_r$ denote $r$ positive integers that are relatively prime in pairs,
and let $a_1,...,a_r\in\mathbb{Z}$. Then the congruences (21) have have common
solutions. If $x_0$ is one such solution, then an integer $x$ satisfies the
congruences (21) iff $x=x_0+km$ for some integer $k$. Here $m-m_1m_2...m_r$

% Lecture 13

\begin{itemize}
    \item $m_1,...,m_r$ positive integers relatively prime in pairs
    \item $m=m_1m_2...m_r$
    \item Instead of considering just one set of aligns (21), we will consider
    all possible systems of this type
    \item Let
    \begin{align}
        \begin{split}
            a_1&\in\{1,...,m_1\}\\
            a_2&\in\{1,...,m_2\}\\
            &...\\
            a_r&\in\{1,...,m_r\}
        \end{split}
    \end{align}
    \item The number of such $r$-tuples $(a_1,...,a_r)$ is $m=m_1m_2...m_r$.
    \item By the \textbf{C.R.T.} each $r$-tuple determines precisely one
    residue class $x$ modulo $m$.
    \begin{itemize}
        \item Moreover, distinct $r$-tuples determine different residue classes.
        To see this, suppose that $(a_1,...,a_r)\neq(a_1',...,a_r')$. then
        $a_i\neq a_i'$ for some $i$, and we see that no integer $x$ satisfies both
        the congruences $x\equiv a_i\ (mod\ m_i)$ and $x\equiv a_i'\ (mod\ m_i)$
    \end{itemize}
    \item This we have a one-to-one correspondence between the $r$-tuples
    $(a_1,...,a_r)$ and a complete residue system modulo $m$,
    such as the integers $1,...,m$
\end{itemize}

\textbf{Theorem 2.19:}\\
If $m_1,\ m_2>0,\ (m_1,m_2)=1$, then $\phi(m_1,m_2)=\phi(m_1)\phi(m_2)$
moreover, if $m=\Pi p^\alpha$ then
\begin{align}
    \phi(m)=\prod_{p|m}(p^\alpha-p^{\alpha-1})=m\prod{p|m}(1-\frac{1}{p})
\end{align}

% Lecture 14

\textbf{Theorem 2.20:}\\
Let $f(x)$ be a fixed polynomial with integral coefficients, and for any positive
integer $m$ let $N(m)$ denote the number of solutions of the congruence
$f(x)\equiv0\ (mod\ m)$. If $m=m_1m_2$ where $(m_1,m_2)=1$, then
$N(m)=N(m_1)N(m_2)$. If $m=\prod p^\alpha$, then $N(m)=\prod N(p^alpha)$

\subsection{Public-key Cryptography}

\textbf{Lemma 2.22:}\\
Suppose $m\in\mathbb{Z},\ m>0,\ (a,m)=1$. If $k,\overline{k}\mathbb{Z}$ and
$k,\overline{k}>0$ such that $k,\overline{k}\equiv1\ (mod\ \phi(m))$, then
$a^{k\overline{k}}\equiv a\ (mod\ m)$.

\textbf{Proof:} Theorem 2.22\\
Write $k\overline{k}=1+r\phi(m)$ for some $r\in\mathbb{Z}$. Then by Euler's congruence
\begin{align*}
    a^{k\overline{k}} = aa^{r\phi(m)} = a(a^{\phi(m)})^r \equiv a\cdot1^r = a\ (mod\ m)
\end{align*}

\begin{itemize}
    \item If $(a,m)=1,\ k>0$, then $(a^k,m)=1$. Thus if $n=\phi(m)$ and
    $r_1,...,r_n$ is a system of reduced residues $(mod\ m)$, then the
    numbers $r_1^k,...,r_n^k$ are also relatively prime to $m$. These $k^\text{th}$
    powers may not all be distinct $(mod\ m)$, as we see by considering the case
    $k=\phi(m)$. On the other hand, from lemma 2.22, we can deduce that
    these $k^\text{th}$ powers are distinct $(mod\ m)$ provided that $(k,\phi(m))=1$.
    \item Suppose that $r_i^k\equiv r_j^k\ (mod\ m)$ and $(k,\phi(m))=1$.
    By theorem 2.9 we may find $\overline{k}>0$ such that
    $k\overline{k}\equiv 1\ (mod\ \phi{m})$ and then it follows from the lemma that
    \begin{align}
        r_i\equiv r_i^{k\overline{k}}=(r_i^k)^{\overline{k}}\equiv (r_j^k)^{\overline{k}}= r_j^{k\overline{k}}\equiv r_j\ (mod\ m)
    \end{align}
    This implies that $i=j$. We will show later that the converse also holds:
    the numbers $r_i^k,...,r_n^k$ are distinct $(mod\ m)$ only if $(k,\phi(m))=1$.
    Suppose that $(k,\phi(m))=1$. Since the numbers $r_1,...,r_n^k$ are distinct
    $(mod\ m)$, they form a system of reduced residues $(mod\ m)$. That is the map
    $a\mapsto a^k$ permutates the reduced residues $(mos\ m)$ if $(k,\phi(m))=1$.
    The significance of the lemma is that the further map $b\mapsto b^{\overline{k}}$
    is the inverse permutation.
    \item To apply these observations to cryptography, we take two distinct large
    primes, $p_1, p_2$, say each one with about 100 digits.
    \begin{itemize}
        \item So $m=p_1p_2$ has about 200 digits.
        \item Since we know the prime factorisation of m, from theorem 2.19
        we have that $\phi(m)=(p_1-1)(p_2-1)$
        \item So $\phi(m)<m$
        \item we choose now a big number $k$, $0<k,\phi(m)$ and check by the
        Euclidean algorithm that $(k,\phi(m))=1$. We try until we get such a $k$.
        \item We make the numbers $m$ and $k$ publicly available, by keep $p_1,p_2$
        and $\phi(m)$ secret.
        \item suppose now thatt some associate of ours wants to send us a message,
        say \textit{'Gauss was a genuis!'}. The associate first converts the
        characters to number in some standard way, say by emplying (ASCII).
        Then $G=071$, $a=097$,..., $!=033$.
        Then concatenate these codes to form a number
        \begin{align*}
            a=071097117115115126119097115126097126103101110105117115033
        \end{align*}
        \item if the message were longer, it could be ficided into a number of blocks.
        \item the associate could send the number $a$ and we could reconstruct the
        message. But suppose that message has some sensitive information. In that
        case the associate would use the number $k$ and $m$ that we have provided.
        \item Our associate quickly finds the unique number $b$, $0\leq b<m$
        such that $b\equiv a^k\ (mod\ m)$ and sends this $b$ to us.
        \item We use Euclidean Algorithm to find $\overline{k}>0$ such that\\
        $k\overline{k}\equiv 1\ (mod\ \phi(m))$ and then we find the unique $c$
        such that $0\leq c<m$, $c\equiv b^{\overline{k}}\ (mod\ m)$.
        From lemma 2.22 we deduce that $a=c$.
    \end{itemize}
    \item In theory it might happen that $(a,m)>1$ in which case the lemma does
    not apply, but the chances of this is $\approx\frac{1}{p_i}\approx 10^{-100}$.
    Suppose that some third party gain access to the numbers $m$, $k$ and $b$,
    and seeks to recover the number a. In principle, all that needs to be done
    is to factor $m$, which yields $\phi(m)$, and hence $\overline{k}$.
    The problem of locating the factors of $m$ for a big number is not easy.
\end{itemize}

% Lecture 15

\subsection{Prime Power Moduli}

Let $f(x)$ be a polynomial with integer coefficients. Let $N(m)$ denote the
number of solutions of $f(x)\equiv 0\ (mod\ m)$. Suppose that $m=m_1m_2$, where
$(m_1,m_2)=$. With a "little work", theorem 2.19 shows that the roots of the
congruence $f(x)\equiv 0\ (mod\ m)$ are in one-to-one correspondence with pairs
$(a_1,a_2)$ in whic $a_1$ runs over all roots of the congruences
$f(x)\equiv 0\ (mod\ m_1and in)$ $a_2$ runs over all roots of the congruence
$f(x)\equiv 0\ (mod\ m_2)$.

\begin{itemize}
    \item From theorem 2.16 and theorem 2.20 we have that the congruence
    $f(x)\equiv 0\ (mod\ m)$ has solutions iff it has solutions $(mod\ p^\alpha)$
    for each prime power $p^\alpha$ exactly dividing $m$.
\end{itemize}

\textbf{Example:}
Let $f(x)=x^2+x+7$. Find all roots of $f(x)\equiv 0\ (mod\ 189)$, given that
$189=3^3\cdot7$, that all roots $(mod\ 27)$ are $4$, $13$, and $22$, and that
the roots $(mod\ 7)$ are $0$ and $6$.

\textbf{Solution:}
By the Eucliean algorithm and (2.2), we find that $x\equiv a_1\ (mod\ 27)$ and that
$x\equiv a_2\ (mod\ 7)$ iff $x\equiv 28a_1-27a_2\ (mod\ 189)$. We let $a_1=4, 13, 22$
and $a_2=0, 6$. Thus we obtain the six solutions $13, 49, 76, 112, 139, 175\ (mod\ 189)$
\begin{itemize}
    \item The problem of solving a congruence is now reduced to the case of a
    prime-power modulus.
    \begin{itemize}
        \item To solve $f(x)\equiv 0\ (mod\ p^k)$ we start with a solutions modulo
        $p$ and then move to $p^2, p^3,...,p^k$.
    \end{itemize}
    Suppose that $x=a$ is a solution of $f(x)\equiv 0\ (mod\ p^j)$ and we want
    to use it to get a solution modulo $p^{j+1}$.
    The idea is to try to get a solution $x=a+tp^j$, where t is to be determined,
    by use of Taylor's expansion
    \begin{align}
        f(a+tp^j)=f(a)+tp^jf'(a)+t^2p^{2j}\frac{f''(a)}{2!}+...+t^np^{nj}\frac{f^{(n)}(a)}{n!}
    \end{align}
    where $n=$ degree of $f(x)$. All derivatives beyond the $n^\text{th}$ are
    identicallly zero. Now with respect to the modulus $p^{j+1}$, equation (37) gives
    \begin{align*}
        f(a+tp^j)\equiv f(a)+tp^jf'(a)\ (mod\ p^{j+1})
    \end{align*}
    as the following argument shows. What we want to establish is that the
    coefficients of $t^1, t^3,..., t^n$ in (37) are divisible by $p^{j+1}$
    and so can be ommited in (38). This is almost obvious because the powers
    of $p$ in those terms. The explanation is that $\frac{f^{(k)}(a)}{k!}$
    is an integer for each value of $k$, $2\leq k\leq n$. To see this, let
    $cx^r$ be a representative term from $f(x)$. The corresponding term in
    $f^{(k)}(a)$ is $cr(r-1)(r-2)...(r-k+1)a^{r-k}$.

    % Lecture 16

    We now use the fact (without proof), that the product of $k$ consecutive
    integers is divisible by $k!$, and the argument is complete.
    Thus, we have proved that the coefficients of $t^2, t^3,..., t^n$ in (37)
    are divisible by $p^{j+1}$. The congruence (38) reveals how $t$ should be
    chosen if $x=a+tp^j$ is to be a solution of $f(x)\equiv 0\ (mod\ p^{j+1})$.
    We want $t$ to be a solution of
    \begin{align}
        f(a)+tp^jf'(a)\equiv 0\ (mod\ p^{j+1})
    \end{align}
    Since $f(x)\equiv 0\ (mod\ p^j)$ have the solutions $x=a$, we see that $p^j$
    can be removed as a factor to given
    \begin{align}
        tf'(a)\equiv -\frac{f(a)}{p^j}\ (mod\ p)
    \end{align}
    Which is a linear congruence in $t$. This congruence may have no solution,
    one solutions, or $p$ solutions. If $f'(a)\equiv 0\ (mod\ p)$, then this
    congruence has exactly one solution, and we obtain
\end{itemize}

\textbf{Theorem 2.3:} Hansel's Lemma:\\
Suppose that $f(x)$ is a polynomial with integral coefficients.
If $f(a)\equiv0\ (mod\ p^j)$ and $f'(a)\not\equiv0\ (mod\ p)$ then there is
a unique $t\ (mod\ p)$ such that $f(a+tp^j)\equiv0\ (mod\ p^{j+1})$

\begin{itemize}
    \item If $f(a)\equiv0\ (mod\ p^j)$, $f(b)\equiv 0\ (mod\ p^k)$, $j<k$ and
    $a\equiv b\ (mod\ p^j)$, then we say that \underline{$b$ lies above $a$}, or
    \underline{$a$ lifts to $b$}.
    \item If $a\equiv b\ (mod\ p^j)$, then $a$ is called a \underline{nonsingular}
    root if $f'(a)\not\equiv 0\ (mod\ p)$; otherwise it is \underline{singular}.
    \item By Hensel's lemma we see that a nonsingular root $a\ (mod\ p)$ lifts
    to a unique root $a_2\ (mod\ p^2)$. Since $a_2\equiv a\ (mod\ p)$ it follows
    by theorem 2.2 that $f'(a_2)\equiv f'(a)\not\equiv 0\ (mod\ p)$. By a second
    application of Hensel's lemma we may lift $a_2$ to form a root $a_3$ of $f(x)$
    modulo $p^3$, and so on.
    \item In general we find that a nonsingular root $a$ modulo $p$ lifts to a
    uniques root $a_j$ modulo $p^j$ ofr $j=2,3,...$ by (2.5) we see that this
    sequence is generated bby means of the recursion
    \begin{align}
        a_{j+1} = a_j-f(a_j)\overline{f'(a)}
    \end{align}
    where $f'(a)$ is an integer chosen so that
    $f'(a)\overline{f'(a)}\equiv 1\ (mod\ p)$.
\end{itemize}

\textbf{Example:}
Solve $x^2+x+47\equiv 0\ (mod\ 7^3)$

\textbf{Solution:}
First we note that $x\equiv 1\ (mod\ 7)$ and $x\equiv 5\ (mod\ 7)$ are the only
solutions of $x^2+x+47\equiv 0\ (mod\ 7)$. Since $f'(x)=2x+1$, we see that
\begin{itemize}
    \item $f'(1)=3\equiv 0\ (mod\ 7)$
    \item $f'(5)=11\equiv 0\ (mod\ 7)$
\end{itemize}
\textit{(So these roots are non singular)}\\
Taking $f'(1)=5$, we see by (40) that the root $a\equiv 1\ (mod\ 7)$ lifts to
$a_2=1$. Since $a_2$ is considered $(mod\ 7^2)$, we may take instead $a_2=1$.
Then $a_3=1-49\cdot5\equiv 99\ (mod\ 7^3)$. Similarly, we take $\overline{f'(5)}=2$
and see by (40) that the root $5\ (mod\ 7)$ lifts to
$5-77\cdot2=-149\equiv 47\ (mod\ 7^2)$ and that $47\ (mod\ 7^2)$ lifts to
$47-f(47)\cdot2=47-2303\cdot2=-4599\equiv 243\ (mod\ 7^3)$.
Thus we conclude that $99$ and $243$ are the desired roots and that there are
no others.

% Lecture 17

\subsection{Prime Modulus}

$f(x)\equiv 0\ (mod\ m)$ $\to$ $f(x)\equiv 0\ (mod\ p)$\quad \textit{(reduced)}
(No general mathod exists to solve such congruences)

\textbf{Question:}\\
Given a polynomial congruence $f(x)\equiv 0\ (mod\ m)$ is there an analogue to
the result in algebra which says that a polynomial equation of degree $n$ with
complex coefficients has exactly $n$ roots?\\
$\to$ for congruences the solution is more complicated.

e.g. For any $m>1$, there are $f(x)$ such that $f(x)\equiv 0\ (mod\ m)$
has no solutions.

e.g.2 $x^p-x+1\equiv 0\ (mod\ m)$, where $p$ is a prime factor of $m$ has no
solutions because $x^p-x+1\equiv 0\ (mod\ p)$ has none, by Fermat's Theorem.

$f(x)=a_nx^n+a_{n-1}x^{n-1}+...+a_1x+a_0$ and we assume $p\nmid a_n$
so that the congruence $f(x)\equiv 0\ (mod\ p)$ has degree $n$.

\textbf{Theorem 2.25:}\\
If the degree $n$ of $f(x)\equiv 0\ (mod\ p)$ is greater than or equal to $p$,
then either every integer is a solution of $f(x)\equiv 0\ (mod\ p)$ or there is
a polynomial $g(x)$ having integral coefficients, with leading coefficient $1$,
such that $g(x)\equiv 0\ (mod\ p)$ is of degree less than $p$ and the solutions
of $g(x)\equiv 0\ (mod\ p)$ are precisely those of $f(x)\equiv 0\ (mod\ p)$.

\textbf{Proof:} Theorem 2.25\\
Dividing $f(x)$ by $x^p-x$ we get a quotient $q(x)$ and a remainder $r(x)$
such that $f(x)=(x^-x)q(x)+r(x)$. here $q(x)$ and $r(x)$ are polynomials
with integral coefficients, and $r(x)=0$ or degree $r(x)<p$. Since every
integer is a solutions of $x^p\equiv x\ (mod\ p)$ are the same as those of
$r(x)\equiv 0\ (mod\ p)$ by Fermat's theorem, we see that the solutions of
$f(x)\equiv 0\ (mod\ p)$ are the same as those of $r(x)\equiv 0\ (mod\ p)$.
If $r(x)=0$ or if every coefficient of $r(x)$ is divisible by $p$, then every
integer is a solution of $f(x)\equiv 0\ (mod\ p)$.

On the other hand, if at least one coefficient of $r(x)$ is not divisible by
$p$, then the congruence $r(x)\equiv 0\ (mod\ p)$ has a degree, and that degree
is  less than $p$. The polynomial $g(x)$ in the theorem can be obtained from
$r(x)$ by getting leading coefficient $1$, as follows. We may discard all terms
in $r(x)$ whose coefficients are divisible by $p$, since the congruence
properties modulo $p$ are unaltered. Then let $bx^m$ be the term of the highest
degree in $r(x)$, with $(b,p)=1$. Choose $\overline{b}$ so that
$b\overline{b}\equiv 1\ (mod\ p)$, and note that $(\overline{b},b)=1$ also. Then
the congruence $\overline{b}r(x)\equiv 0\ (mod\ p)$ has the same solutions as
$r(x)\equiv 0\ (mod\ p)$, and so has the same solutions as $f(x)\equiv 0\ (mod\ p)$.
Define $g(x)=\overline{b}r(x)$ with its leading coefficient $b\overline{b}$
replaced by 1, that is,
\begin{align}
    g(x)=\overline{b}r(x)-(b\overline{b}-1)x^m
\end{align}

\textbf{Theorem 2.26:}\\
The congruence $f(x)\equiv 0\ (mod\ p)$ of degree $n$ has at most $n$ solutions.

\textbf{Proof:} Theorem 2.26\\
The proof is by induction on the degree of $f(x)\equiv 0\ (mod\ p)$.
If $n=0$, the polynomial $f(x)=a_0$ with $a_0\not\equiv 0\ (mod\ p)$ and hence
the congruence has no solutions. If $n=1$, the congruence has exactly one
solutions by theorem 2.17.
Assume the truth of the theorem for all congruences of degree $<n$, supppose that
there were more than $n$ solutions of the congruence $f(x)\equiv 0\ (mod\ p)$ of
degree $n$. Let the leading term of $f(x)$ be $a_nx^n$ and let $u_1,...,u_{n+1}$
be solutions of the congruence with $u_1\not\equiv u_j\ (mod\ p)$ for $i\neq j$.
We define $g(x)$ by
\begin{align}
    g(x)=f(x)-a_n(x-u_1)...(x-u_n)
\end{align}
noting the cancellation of $a_nx^n$ on the right.

Note that $g(x)\equiv 0\ (mod\ p)$ has at least $n$ solutions, namely $u_1,...,u_n$.
We cansider two cases:
\begin{enumerate}
    \item every coefficient. of $g(x)$ is divisible by $p$
    \item at least one coefficient is not divisible by $p$
\end{enumerate}

For (i), every integer is a solution of $g(x)\equiv 0\ (mod\ p)$, and since
$f(u_{n+1})\equiv 0\ (mod\ p)$ by assumption, it follows that $x=u_{n+1}$ is a
solutions of
\begin{align}
    a_n(x-u_1)...(x-u_n)\equiv 0\ (mod\ p)
\end{align}
This contradicts theorem 1.15.

For (ii), we note that $g(x)\equiv 0\ (mod\ p)$ has a degree and that degree is
$<n$. By the induction hypothesis, this congruence has fewer than n solutions. This
contradicts the earlier observation that this congruence has at least $n$ solutions.
Thus the proof is complete.

\textbf{Corollary 2.27:}
If $b_nx^n+b_{n-1}x^{n-1}+...+b_0\equiv 0\ (mod\ p)$ has more than $n$ solutions,
then all the coefficients $b_j$ are divisible by $p$.

\textbf{Theorem 2.28:}\\
If $F(x)$ is a function that maps residue classes $(mod\ p)$ to residue classes
$(mod\ p)$, then there is a polynomial $f(x)$ with integral coefficients and
degree at most $p-1$ such that $f(x)\equiv F(x)\ (mod\ p)$ for all residue
classes $x\ (mod\ p)$.

\textbf{Theorem 3.2:} Gauss' Lemma\\
Let $p$ be an odd prime and $(a,p)=1$.
\begin{align}
    a,2a,3a,...,\frac{p-1}{2}a
\end{align}
and their least positive residues









\end{document}
