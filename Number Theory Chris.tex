\documentclass[a4paper]{article}

\usepackage{amsmath}
\usepackage{amsfonts}
\usepackage{amssymb}
\usepackage{enumitem}
\setlength{\parindent}{0em}
\setlength{\parskip}{1em}
\renewcommand{\baselinestretch}{1}
\title{Number Theory: Lecture Notes}
\author{Anthony Dunford & Chris Nash}
\begin{document}
\maketitle

\section{Divisibility and Primes}

\subsection{Introduction}

\textbf{Well ordering Principle:}\\
Let $S\neq0$ be a set of positive integers.\\
Then there exists $s\in S$ such that for all $a\in S, s\leq a$

\textbf{Induction:}\\
If a set $s$ of positive integers contains the integer $1$\\
And contains $n+1$ whenever it contains $n$\\
Then $S$ consists of all the positive integers

\subsection{Divisibility}

\textbf{Definition 1.1:} Divisibility\\
An integer $b$ is divisible by and integer $a\neq0$ if there is an integer
$x$ such that $b=ax$.\\s
We write $a|b$ (a divides b)

\textbf{Theorem 1.1:} Properties of divisibility
\begin{enumerate}
    \item $a|b\            \rightarrow a|bc\quad c\in \mathbb{Z}$
    \item $a|b\ \&\ b|c    \rightarrow a|c$
    \item $a|b\ \&\ a|c    \rightarrow a|(bx+cy)\quad x,y\in\mathbb{Z}$
    \item $a|b\ \&\ b|a    \rightarrow a=\pm b$
    \item $a|b,\ a>0,\ b>0 \rightarrow a\leq b$
    \item $m\neq0,\ a|b\   \leftrightarrow ma|mb$
\end{enumerate}

\textbf{Proof:} Theorem 1.1 (3)

$a|b \rightarrow b=ar$ for some $r\in \mathbb{Z}$
and $a|c \rightarrow c=as$ for some $s\in \mathbb{Z}$
Hence $bx+cy=a(rx+sy)$ and this proves that $a|(bx+cy)$


\textbf{Theorem 1.2:} The Division Algorithm\\
Let $a,b\in\mathbb{Z},\ a>0$.\\
Then there exists unique $q,r\in\mathbb{Z}$ such that $b=qa+r,\ 0\leq r<a$.\\
If $a\nmid b$ then $0<r<a$

\textbf{Proof:} Theorem 1.2

Consider the arithmetic progression:

$...,b-3a,b-2a,b-a,b,b+a,b+2a,b+3a,...$

In the sequence select the smallest non-negative member and denote it by
$r$. Thus by definition $r$ satisfies the inequalities of the theorem. But
also $r$, being in the sequence, is of the form $b-qa$, and thus q is defined
in terms of $r$.

To prove uniqueness we suppose there is another pair $q_1$ and $r_1$ satisfying
the same conditions. First we prove that $r=r_1$. If not, we may presume
that $r<r_1$ so that $=<r_1-r<a$ and then we see that $r_1-r=a(q-q_1)$ and
so $a|(r_1-r)$, a contradiction to Theorem 1.1 (5). Hence $r=r_1$ and also
$q=q_1$.

Note: We stated the theorem with $a>0$. However this is not necessary and
we may formulate as:

Given $a,b\in\mathbb{Z}$ , $a\neq 0$ , there exists $q,r\in\mathbb{Z}$ such
that $b=qa+r$ , $0\leq r < |a|$.


\textbf{Definition 1.2:}\\
The integer $a$ is a \underline{common divisor} of $b$ and $c$ if $a|b$,
$a|c$
and at least $b\neq0$ or $c\neq0$, the greatest among their common divisors
is
called the \underline{greatest common divisor} of $b$ and $c$ and is denoted
by
$gcd(b,c)$ or $(b,c)$.

Let $b_1,...,b_n\in\mathbb{Z}$, not all zero.
We denote $g=(b_1,...b_n)$ to be the greatest common divisor.

\textbf{Theorem 1.3:}\\
If $g=(b,c)$, then there exist $x_0,y_0\in\mathbb{Z}$ such that $g=(b,c)=bx_0+cy_0$

\textbf{Proof:} Theorem 1.3

Consider the linear combination $bx+cy$, where $x,y$ range over all the integers.
This set of integers \{$bx+cy$\} includes positive and negative values and
also 0. ($x=y=0$). Choose $x_0$ and $y_0$ so that $bx_0+cy_0$ is the least
positive integer $l$ in the set. Thus $l=bx_0+cy_0$.

Next we prove that $l|b$ and $l|c$. Assume that $l \nmid b$ , then it follows
that there exists integers $q$ and $r$ , by Theorem 1.2, such that $b=lq+r$
with $0<r<l$. Hence we have $r=b-lq=b-q(bx_0+cy_0)=b(l-qx_0)+c(-qy_0)$, and
thus $r$ is in the set \{$bx+cy$\}. This contradicts the fact that $l$ is
the least positive integer in \{$bx+cy\}$. Similar proof for $l|c$. Now since
$g=(b,c)$ we may write $b=gB$ , $c=gC$ and $l=bx_0+cy_0=g(Bx_0+Cy_0)$. Thus
$g|l $ and so by Theorem 1.1 (5) we conclude that $g \leq l$. We know $g<l$
is impossible since $g$ is the greatest common divisor, so $g=l=bx_0+cy_0$.

\textbf{Theorem 1.4:}\\
The greatest common denominator of $b$ and $c$ can be characterised in the
following two ways:
\begin{enumerate}
    \item It is the least positive value of $bx+cy$ where $x,\ y\in\mathbb{Z}$
    \item If $d$ is any common divisor of $b$ and $c$ then $d|g$ by Theorem
1.1 (3).
\end{enumerate}

\textbf{Proof:} Theorem 1.4

\begin{enumerate}
\item Follows from Theorem 1.3
\item If $d$ is any common divisor of $b$ and $c$, then $d|g$ by Theorem
1.1 (3). Moreover, there cannot be two distinct integers with property (2),
because of Theorem 1.1 (4).
\end{enumerate}

Note: If $d=bx+cy$ , then $d$ is not necessary the $gcd(b,c)$. However, it
does follow from such align* that $(b,c)$ is a divisor of $d$. In particular
, if $bx+cy=1$ for some $x,y\in\mathbb{Z}$ , then $(b,c)=1$.

\textbf{Theorem 1.5:}\\
Given $b_1,...,b_n\in\mathbb{Z}$ not all zero with greatest common divisor
$g$,
there exists integers $x_1,...,x_n$,  such that
\begin{align*}
    g=(b_1,...,b_n)=\sum^n_{j=1}b_jx_j
\end{align*}
Furthermore, g is the least positive value of the linear form $\sum^n_{j=i}b_jy_j$
where the $y_j$ runs over all integers; also $g$ is the positive common divisor
of $b_1,...,b_n$
that is divisible by every common divisor.

\textbf{Proof:} Theorem 1.5

Exercise for the reader.



\textbf{Theorem 1.6:}\\
For any $m\in\mathbb{Z}, m>0$
\begin{align*}
    (ma,mb)=m(a,b)
\end{align*}

\textbf{Proof:} Theorem 1.6

By Theorem 1.4 we have:

$(ma,mb)$ = least positive value of $max+mby$
=$m$ \{ least positive integer of $ax+by$\}
= $m(a,b)$


\textbf{Theorem 1.7:}\\
If $d|a$, $d|b$ and $d>0$, then
\begin{align*}
    \bigg(\frac{a}{d},\frac{b}{d}\bigg)=\frac{1}{d}(a,b)
\end{align*}
If $(a,b)=g$, then
\begin{align*}
    \bigg(\frac{a}{g},\frac{b}{g}\bigg)=1
\end{align*}



\textbf{Proof:} Theorem 1.7

The second assertion is the special case of the first using $d=(a,b)=g$.
The first assertion is a direct consequence of Theorem 1.6, obtained by replacing
$m,a,b$ in Theorem 1.6 by $d,\frac{a}{d},\frac{b}{d}$ respectively.

\textbf{Theorem 1.8:}\\
If $(a,m)=(b,m)=1$ then $(ab,m)=1$

\textbf{Proof:} Theorem 1.8

Exercise for the reader.

\textbf{Definition:} 1.3

We say that $a$ and $b$ are \underline{relatively prime} in case $(a,b)=1$
, and that $a_1,a_2,...,a_n$ are relatively prime in the case $(a_1,a_2,...,a_n)=1$.
We say that $a_1,a_2,...,a_n$ are \underline{relatively prime in pairs} in
case $(a_i,a_j)=1$ for all $i=1,2,...,n$ and $j=1,2,...n$ with $i \neq j$.

Note: $(a,b)=1$ we also say $a$ and $b$ are \underline{coprime}.

\textbf{Theorem 1.9:}\\
For any $x\in\mathbb{Z}$ we have
\begin{align*}
(a,b)=(b,a)=(a,-b)=(a,b+ax)
\end{align*}

\textbf{Proof:} Theorem 1.9

Exercise for the reader.

\textbf{Theorem 1.10:} Euclid's Lemma\\
If $c|ab$ and $(b,c)=1$, then $c|a$.

\textbf{Proof:} Theorem 1.10

By Theorem 1.6 , $(ab,ac)=a(b,c)=a$. By hypothesis $c|ab$ and clearly $c|ac$,
so $c|a$ by Theorem 1.4 (2).

Now we observe for $c\neq 0$ , we have $(b,c)=(b,-c)$ by Theorem 1.9 and
hence we may presume $c>0$. \\

\textbf{Theorem 1.11:} The Euclidean Algorithm\\
Given $b,c\in\mathbb{Z}, c>0$, we can make a repeated
application of the division algorithm, \textbf{Theorem 1.2},
to obtain a series of align*s
\begin{align}
b=cq_1+r_1          & \quad0<r_1<c\\
c=r_1q_2+r_2        & \quad0<r_2<r_1\\
r_1=r_2q_3+r_3      & \quad0<r_3<r_2\\
...\\
r_j=r_{j+1}q_j+r_j  & \quad0<r_j<r_{j-1}\\
r_{j-1}=r_jq_{j+1}.
\end{align}
The greatest common divisor $(b,c)$ of $b$ and $c$ is $r_j$,
the last nonzero remainder in the division process.
Values of $x_0$ and $y_0$ in $(b,c)=bx_0+cy_0$
can be obtained by writing each $r_i$ as a linear combination
of $b$ and $c$.

\textbf{Proof:} Theorem 1.11

See Theorem 1.11 in the textbook or Theorem 1.13 in the Lecture Notes.

\textbf{Example 1}
$gcd(841,160)$
\begin{align*}
\begin{split}
841&=160\times5 + 41 \\
160&=41\times3 + 37 \\
41&=37\times 1 + 4 \\
37&=34\times 9 + 1 \\
4&=1\times 4 + 0 \\
\end{split}
\end{align*}

Hence (841,160)=1 working backwards gives:


\begin{align*}
\begin{split}
1=37\times1 - 4\times9 \\
1=37\times1 - (41-37)\times9 \\
1=37\times10 - 41\times9 \\
1=(160-3\times41)\times10 - 41 \times 9 \\
1=160\times10 - 41\times39 \\
1=160\times10 - (841-160\times5)\times39 \\
1=(-39)\times841 + 205\times160 \\
\end{split}
\end{align*}

Note the solution is not unique:
\begin{align*}
1=121\times841 - 636\times 160
\end{align*}

\textbf{Example 2} Extended Algorithm


\begin{align*}
\begin{split}
r_i&=r_{i-2} - q_ir_{i-1} \\
x_i&=x_{i-2} - q_ix_{i-1} \\
y_i&=y_{i-2} - q_iy_{i-1} \\
r_1&=b , r_0=c \\
x_1&=1 , x_0=0 \\
y_1&=0 , y_0=1 \\
\end{split}
\end{align*}



We want to compute the $gcd(841,160)$ and express as a linear combination
of 841 and 160.


\textbf{Definition 1.4:}\\
The integers $a_1,...,a_n$, all different from zero, have a
\textbf{common multiple} $b$ if $a_i|b$ for $i=1,...,n$. The least of
the positive common multiples is called the
\textbf{least common multiple} and it is denoted by
$[a_1,...,a_n]$ or $lcm(a1,...,a_n)$

\textbf{Theorem 1.12:}\\
If $b$ is any common multiple of $a_1,...,a_n$,
then $[a_1,...,a_n]\ |\ b$. This is the same as saying that if $h=[a_1,...,a_n]$
then $0,\pm h,2\pm h,...$ comprise all the common multiples of $a_1,...,a_n$.

\textbf{Proof:} Theorem 1.12

Let $m$ be any common multiple and divide $m$ and $h$. By Theorem 1.2 , $\exists
q,r$ such that $m=qh+r$ , $ 0\leq r<h.$ We must probe that $r=0$. If $r\neq0$
we argue as follows. For each $i=1,2,...,n$ we know that $a_i|h$ and $a-i|m$
, so that $a_i|r$ . Thus $r$ is a positive common multiple of $a_1,a_2,...,a_n$
contrary to the fact that h is the least of all positive common multiples.


\textbf{Theorem 1.13:}\\
If $m>0$
\begin{enumerate}
    \item $[ma,mb] = m[a,b]$
    \item $[a,b](a,b)=|ab|$
\end{enumerate}

\textbf{Proof:} Theorem 1.13

\begin{enumerate}
\item Let $H=[ma,mb]$ and $h=[a,b]$. Then $mh$ is a multiple of $ma$ and
$mb$, so that $mh\geq H$. Also, $H$ is a multiple of both $ma$ and $mb$ so
$H/m$ is a multiple of $a$ and $b$. Thus, $H/m \geq h$ from which it allows
that $mh=H$.

\item It will suffice to prove this for $a,b\in\mathbb{Z}$ with $a>0,b>0$
, since $[a,-b]=[a,b]$. We begin with the special case where $(a,b)=1$. Now
$[a,b]=1$, is a multiple of a, say $ma$. Then $b|ma$ and $(a,b)=1$, so by
Theorem 1.10 we conclude that $b|m$. Hence $b \leq m$ , $ba\leq ma$. But
$ba$, being a positive common multiple of 4b4 and $a$ , cannot be less tahn
the least common multiple, so $ba=ma=[a,b]$.

Let $(a,b)=g>1$. we have $(a/g,b/g)=1$ by Theorem 1.7. Applying the result
of the previous paragraph we have:
\begin{align*}
[a/g,b/g]\cdot(a/g,b/g)=ab/g
\end{align*}

Multiplying by $g^2$ and using Theorem 1.6 as well as the first part (1.),
we get $[a,b]\cdot(a,b)=ab$.
\end{enumerate}



\end{document}
